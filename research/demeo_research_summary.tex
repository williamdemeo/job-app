\begin{filecontents*}{inputs/refs2.bib}
@article{MR2463991,
    AUTHOR = {Gonthier, Georges},
     TITLE = {Formal proof---the four-color theorem},
   JOURNAL = {Notices Amer. Math. Soc.},
  FJOURNAL = {Notices of the American Mathematical Society},
    VOLUME = {55},
      YEAR = {2008},
    NUMBER = {11},
     PAGES = {1382--1393},
      ISSN = {0002-9920},
     CODEN = {AMNOAN},
   MRCLASS = {05C15 (68T15)},
  MRNUMBER = {2463991 (2009j:05079)},
MRREVIEWER = {Solomon Marcus},
}
@book{MR769301,
    AUTHOR = {Martin-L{\"o}f, Per},
     TITLE = {Intuitionistic type theory},
    SERIES = {Studies in Proof Theory. Lecture Notes},
    VOLUME = {1},
      NOTE = {Notes by Giovanni Sambin},
 PUBLISHER = {Bibliopolis, Naples},
      YEAR = {1984},
     PAGES = {iv+91},
      ISBN = {88-7088-105-9},
   MRCLASS = {03B15 (03F50 03F55)},
  MRNUMBER = {769301 (86j:03005)},
MRREVIEWER = {M. M. Richter},
}
@Manual{Coq:manual,
  title =        {The Coq proof assistant reference manual},
  author =       {The Coq development team},
  organization = {LogiCal Project},
  note =         {Version 8.0},
  year =         {2004},
  url =          "http://coq.inria.fr"
}
@book{MR2229784,
    AUTHOR = {Bertot, Yves and Cast{\'e}ran, Pierre},
     TITLE = {Interactive theorem proving and program development},
    SERIES = {Texts in Theoretical Computer Science. An EATCS Series},
      NOTE = {Coq'Art: the calculus of inductive constructions,
              With a foreword by G{\'e}rard Huet and Christine
              Paulin-Mohring},
 PUBLISHER = {Springer-Verlag, Berlin},
      YEAR = {2004},
     PAGES = {xxvi+469},
      ISBN = {3-540-20854-2},
   MRCLASS = {68-01 (03B35 68T15)},
  MRNUMBER = {2229784 (2007i:68001)},
       DOI = {10.1007/978-3-662-07964-5},
       URL = {http://dx.doi.org/10.1007/978-3-662-07964-5},
}
@article{MR935892,
    AUTHOR = {Coquand, Thierry and Huet, G{\'e}rard},
     TITLE = {The calculus of constructions},
   JOURNAL = {Inform. and Comput.},
  FJOURNAL = {Information and Computation},
    VOLUME = {76},
      YEAR = {1988},
    NUMBER = {2-3},
     PAGES = {95--120},
      ISSN = {0890-5401},
   MRCLASS = {68Q55 (03B15 03B35 03B40 03B70)},
  MRNUMBER = {935892 (89j:68096)},
MRREVIEWER = {R. P. Nederpelt},
       DOI = {10.1016/0890-5401(88)90005-3},
       URL = {http://dx.doi.org/10.1016/0890-5401(88)90005-3},
}
@article {Palfy:1980,
    AUTHOR = {P{\'a}lfy, P{\'e}ter P{\'a}l and Pudl{\'a}k, Pavel},
     TITLE = {Congruence lattices of finite algebras and intervals in
              subgroup lattices of finite groups},
   JOURNAL = {Algebra Universalis},
  FJOURNAL = {Algebra Universalis},
    VOLUME = {11},
      YEAR = {1980},
    NUMBER = {1},
     PAGES = {22--27},
      ISSN = {0002-5240},
   MRCLASS = {08A30 (06B15 08A60 20E15)},
  MRNUMBER = {593011 (82g:08003)},
MRREVIEWER = {Ralph Freese},
DOI = {10.1007/BF02483080}
}
@article {MR3076179,
    AUTHOR = {Kearnes, Keith A. and Kiss, Emil W.},
     TITLE = {The shape of congruence lattices},
   JOURNAL = {Mem. Amer. Math. Soc.},
  FJOURNAL = {Memoirs of the American Mathematical Society},
    VOLUME = {222},
      YEAR = {2013},
    NUMBER = {1046},
     PAGES = {viii+169},
      ISSN = {0065-9266},
      ISBN = {978-0-8218-8323-5},
   MRCLASS = {08B05 (08B10)},
  MRNUMBER = {3076179},
MRREVIEWER = {James B. Nation},
       DOI = {10.1090/S0065-9266-2012-00667-8},
       URL = {http://dx.doi.org/10.1090/S0065-9266-2012-00667-8},
}
  @article {Freese:2009,
    AUTHOR = {Freese, Ralph and Valeriote, Matthew A.},
    TITLE = {On the complexity of some {M}altsev conditions},
    JOURNAL = {Internat. J. Algebra Comput.},
    FJOURNAL = {International Journal of Algebra and Computation},
    VOLUME = {19},
    YEAR = {2009},
    NUMBER = {1},
    PAGES = {41--77},
    ISSN = {0218-1967},
    MRCLASS = {08B05 (03C05 08B10 68Q25)},
    MRNUMBER = {2494469 (2010a:08008)},
    MRREVIEWER = {Clifford H. Bergman},
    DOI = {10.1142/S0218196709004956},
    URL = {http://dx.doi.org/10.1142/S0218196709004956}
  }
@book {MR2839398,
    AUTHOR = {Bergman, Clifford},
     TITLE = {Universal algebra},
    SERIES = {Pure and Applied Mathematics (Boca Raton)},
    VOLUME = {301},
      NOTE = {Fundamentals and selected topics},
 PUBLISHER = {CRC Press, Boca Raton, FL},
      YEAR = {2012},
     PAGES = {xii+308},
      ISBN = {978-1-4398-5129-6},
   MRCLASS = {08-02 (06-02 08A40 08B05 08B10 08B26)},
  MRNUMBER = {2839398 (2012k:08001)},
MRREVIEWER = {Konrad P. Pi{\'o}ro},
}

@unpublished{Bergman-DeMeo:2016,
    AUTHOR = {Bergman, Clifford and DeMeo, William},
    TITLE = {Universal Algebraic Methods for Constraint Satisfaction Problems},
    YEAR = {2016},
    NOTE = {Submitted to {L}{M}{C}{S}},
    URL = {https://arxiv.org/abs/1611.02867}
}
@article {MR3350338,
    AUTHOR = {Bergman, Clifford and Failing, David},
    TITLE = {Commutative idempotent groupoids and the constraint
      satisfaction problem},
    JOURNAL = {Algebra Universalis},
    FJOURNAL = {Algebra Universalis},
    VOLUME = {73},
    YEAR = {2015},
    NUMBER = {3-4},
    PAGES = {391--417},
    ISSN = {0002-5240},
    MRCLASS = {08A70 (08B25 68Q25)},
    MRNUMBER = {3350338},
    DOI = {10.1007/s00012-015-0323-6},
    URL = {http://dx.doi.org/10.1007/s00012-015-0323-6},
  }
@book {Neumann:1967,
    AUTHOR = {Neumann, Hanna},
     TITLE = {Varieties of groups},
 PUBLISHER = {Springer-Verlag New York, Inc., New York},
      YEAR = {1967},
     PAGES = {x+192},
   MRCLASS = {20.08 (08.00)},
  MRNUMBER = {0215899 (35 \#6734)},
MRREVIEWER = {I. D. Macdonald},
}
@preamble{
   "\def\cprime{$'$} "
}
@article {Malcev:1967,
    AUTHOR = {Mal{\cprime}cev, A. I.},
     TITLE = {Multiplication of classes of algebraic systems},
   JOURNAL = {Sibirsk. Mat. \v Z.},
  FJOURNAL = {Akademija Nauk SSSR. Sibirskoe Otdelenie. Sibirski\u\i\
              Matemati\v ceski\u\i\ \v Zurnal},
    VOLUME = {8},
      YEAR = {1967},
     PAGES = {346--365},
      ISSN = {0037-4474},
   MRCLASS = {08.30},
  MRNUMBER = {0213276 (35 \#4140)},
MRREVIEWER = {R. S. Pierce},
}
@unpublished{TreeOnTop,
    AUTHOR = {Mar{\'o}ti, Mikl{\'o}},
     TITLE = {The calculus of constructions},
   JOURNAL = {Inform. and Comput.},
  FJOURNAL = {Information and Computation},
    VOLUME = {76},
      YEAR = {1988},
    NUMBER = {2-3},
     PAGES = {95--120},
      ISSN = {0890-5401},
   MRCLASS = {68Q55 (03B15 03B35 03B40 03B70)},
  MRNUMBER = {935892 (89j:68096)},
MRREVIEWER = {R. P. Nederpelt},
       DOI = {10.1016/0890-5401(88)90005-3},
       URL = {http://dx.doi.org/10.1016/0890-5401(88)90005-3},
}
@article{MR911575,
    AUTHOR = {Szendrei, {\'A}gnes},
     TITLE = {Idempotent algebras with restrictions on subalgebras},
   JOURNAL = {Acta Sci. Math. (Szeged)},
  FJOURNAL = {Acta Universitatis Szegediensis. Acta Scientiarum
              Mathematicarum},
    VOLUME = {51},
      YEAR = {1987},
    NUMBER = {1-2},
     PAGES = {251--268},
      ISSN = {0001-6969},
     CODEN = {ASMCBP},
   MRCLASS = {08A40},
  MRNUMBER = {911575 (89d:08005)},
MRREVIEWER = {David M. Clark}
}
@article{MR2333368,
    AUTHOR = {Kearnes, Keith A. and Tschantz, Steven T.},
     TITLE = {Automorphism groups of squares and of free algebras},
   JOURNAL = {Internat. J. Algebra Comput.},
  FJOURNAL = {International Journal of Algebra and Computation},
    VOLUME = {17},
      YEAR = {2007},
    NUMBER = {3},
     PAGES = {461--505},
      ISSN = {0218-1967},
   MRCLASS = {08A35 (08B20 20B25)},
  MRNUMBER = {2333368 (2008f:08003)},
MRREVIEWER = {Giovanni Ferrero},
       DOI = {10.1142/S0218196707003615},
       URL = {http://dx.doi.org/10.1142/S0218196707003615}
}
@BOOK{alvi:1987,
    AUTHOR = {McKenzie, Ralph N. and McNulty, George F. and Taylor, Walter F.},
     TITLE = {Algebras, lattices, varieties. {V}ol. {I}},
 PUBLISHER = {Wadsworth \& Brooks/Cole},
   ADDRESS = {Monterey, CA},
      YEAR = {1987},
     PAGES = {xvi+361},
      ISBN = {0-534-07651-3},
   MRCLASS = {08-01 (06-01)},
  MRNUMBER = {883644 (88e:08001)},
MRREVIEWER = {Gudrun Kalmbach},
}
@article{MR2926316,
    AUTHOR = {Markovi{\'c}, Petar and Mar{\'o}ti, Mikl{\'o}s and McKenzie,
              Ralph},
     TITLE = {Finitely related clones and algebras with cube terms},
   JOURNAL = {Order},
  FJOURNAL = {Order. A Journal on the Theory of Ordered Sets and its
              Applications},
    VOLUME = {29},
      YEAR = {2012},
    NUMBER = {2},
     PAGES = {345--359},
      ISSN = {0167-8094},
     CODEN = {ORDRE5},
   MRCLASS = {08A62 (08A99 08B05 08B10)},
  MRNUMBER = {2926316},
MRREVIEWER = {Anna Romanowska},
       DOI = {10.1007/s11083-011-9232-2},
       URL = {http://dx.doi.org/10.1007/s11083-011-9232-2},
}
@article{MR1630445,
    AUTHOR = {Feder, Tom{\'a}s and Vardi, Moshe Y.},
     TITLE = {The computational structure of monotone monadic {SNP} and
              constraint satisfaction: a study through {D}atalog and group
              theory},
   JOURNAL = {SIAM J. Comput.},
  FJOURNAL = {SIAM Journal on Computing},
    VOLUME = {28},
      YEAR = {1999},
    NUMBER = {1},
     PAGES = {57--104 (electronic)},
      ISSN = {0097-5397},
   MRCLASS = {68Q15 (03D15 68R05)},
  MRNUMBER = {1630445 (2000e:68063)},
MRREVIEWER = {A. M. Dawes},
       DOI = {10.1137/S0097539794266766},
       URL = {http://dx.doi.org/10.1137/S0097539794266766},
}
@article{MR2137072,
    AUTHOR = {Bulatov, Andrei and Jeavons, Peter and Krokhin, Andrei},
     TITLE = {Classifying the complexity of constraints using finite
              algebras},
   JOURNAL = {SIAM J. Comput.},
  FJOURNAL = {SIAM Journal on Computing},
    VOLUME = {34},
      YEAR = {2005},
    NUMBER = {3},
     PAGES = {720--742},
      ISSN = {0097-5397},
     CODEN = {SMJCAT},
   MRCLASS = {68T20 (08A70 68Q25)},
  MRNUMBER = {2137072 (2005k:68181)},
MRREVIEWER = {Benoit Larose},
       DOI = {10.1137/S0097539700376676},
       URL = {http://dx.doi.org/10.1137/S0097539700376676},
}
@article{MR0464698,
    AUTHOR = {Ladner, Richard E.},
     TITLE = {On the structure of polynomial time reducibility},
   JOURNAL = {J. Assoc. Comput. Mach.},
  FJOURNAL = {Journal of the Association for Computing Machinery},
    VOLUME = {22},
      YEAR = {1975},
     PAGES = {155--171},
      ISSN = {0004-5411},
   MRCLASS = {68A20},
  MRNUMBER = {0464698 (57 \#4623)},
MRREVIEWER = {Pavel Strnad},
}
@article {MR2416347,
    AUTHOR = {Cohen, David and Jeavons, Peter and Gyssens, Marc},
     TITLE = {A unified theory of structural tractability for constraint
              satisfaction problems},
   JOURNAL = {J. Comput. System Sci.},
  FJOURNAL = {Journal of Computer and System Sciences},
    VOLUME = {74},
      YEAR = {2008},
    NUMBER = {5},
     PAGES = {721--743},
      ISSN = {0022-0000},
     CODEN = {JCSSBM},
   MRCLASS = {68Q25 (68T20 90B80)},
  MRNUMBER = {2416347},
       DOI = {10.1016/j.jcss.2007.08.001},
       URL = {http://dx.doi.org/10.1016/j.jcss.2007.08.001},
}
@article {MR1481313,
    AUTHOR = {Jeavons, Peter and Cohen, David and Gyssens, Marc},
     TITLE = {Closure properties of constraints},
   JOURNAL = {J. ACM},
  FJOURNAL = {Journal of the ACM},
    VOLUME = {44},
      YEAR = {1997},
    NUMBER = {4},
     PAGES = {527--548},
      ISSN = {0004-5411},
   MRCLASS = {68Q25 (68Q15 68R05 68T20)},
  MRNUMBER = {1481313 (99a:68089)},
MRREVIEWER = {Armin Cremers},
       DOI = {10.1145/263867.263489},
       URL = {http://dx.doi.org/10.1145/263867.263489},
}
@article {MR2563736,
  AUTHOR = {Berman, Joel and Idziak, Pawe{\l} and Markovi{\'c}, Petar and
      McKenzie, Ralph and Valeriote, Matthew and Willard, Ross},
    TITLE = {Varieties with few subalgebras of powers},
    JOURNAL = {Trans. Amer. Math. Soc.},
    FJOURNAL = {Transactions of the American Mathematical Society},
    VOLUME = {362},
    YEAR = {2010},
    NUMBER = {3},
    PAGES = {1445--1473},
    ISSN = {0002-9947},
    CODEN = {TAMTAM},
    MRCLASS = {08B05 (08A30 08A70 08B10 68Q25 68Q32 68T20)},
    MRNUMBER = {2563736 (2010k:08010)},
    MRREVIEWER = {Ivan Chajda},
    DOI = {10.1090/S0002-9947-09-04874-0},
    URL = {http://dx.doi.org/10.1090/S0002-9947-09-04874-0}
  }
  @incollection {MR2648455,
    AUTHOR = {Barto, Libor and Kozik, Marcin},
    TITLE = {Constraint satisfaction problems of bounded width},
    BOOKTITLE = {2009 50th {A}nnual {IEEE} {S}ymposium on {F}oundations of
              {C}omputer {S}cience ({FOCS} 2009)},
    PAGES = {595--603},
    PUBLISHER = {IEEE Computer Soc., Los Alamitos, CA},
    YEAR = {2009},
    MRCLASS = {68Q25 (68T20)},
    MRNUMBER = {2648455 (2011d:68051)},
    DOI = {10.1109/FOCS.2009.32},
    URL = {http://dx.doi.org/10.1109/FOCS.2009.32},
}
@incollection {MR2953899,
    AUTHOR = {Barto, Libor and Kozik, Marcin},
     TITLE = {New conditions for {T}aylor varieties and {CSP}},
 BOOKTITLE = {25th {A}nnual {IEEE} {S}ymposium on {L}ogic in {C}omputer
              {S}cience {LICS} 2010},
     PAGES = {100--109},
 PUBLISHER = {IEEE Computer Soc., Los Alamitos, CA},
      YEAR = {2010},
   MRCLASS = {08A70 (68Q25)},
  MRNUMBER = {2953899},
}
  @article {MR2893395,
    AUTHOR = {Barto, Libor and Kozik, Marcin},
    TITLE = {Absorbing subalgebras, cyclic terms, and the constraint
      satisfaction problem},
    JOURNAL = {Log. Methods Comput. Sci.},
    FJOURNAL = {Logical Methods in Computer Science},
    VOLUME = {8},
    YEAR = {2012},
    NUMBER = {1},
    PAGES = {1:07, 27},
    ISSN = {1860-5974},
    MRCLASS = {68Q17 (08A70)},
    MRNUMBER = {2893395},
    DOI = {10.2168/LMCS-8(1:7)2012},
    URL = {http://dx.doi.org/10.2168/LMCS-8(1:7)2012}
  }
  @article {MR3374664,
    AUTHOR = {Barto, Libor and Kozik, Marcin and Stanovsk{\'y}, David},
    TITLE = {Mal'tsev conditions, lack of absorption, and solvability},
    JOURNAL = {Algebra Universalis},
    FJOURNAL = {Algebra Universalis},
    VOLUME = {74},
    YEAR = {2015},
    NUMBER = {1-2},
    PAGES = {185--206},
    ISSN = {0002-5240},
    MRCLASS = {08B05 (08A05)},
    MRNUMBER = {3374664},
    DOI = {10.1007/s00012-015-0338-z},
    URL = {http://dx.doi.org/10.1007/s00012-015-0338-z},
  }  
@misc{william_demeo_2016_53936,
  author       = {William DeMeo and
                  Ralph Freese},
  title        = {AlgebraFiles v1.0.1},
  month        = May,
  year         = 2016,
  doi          = {10.5281/zenodo.53936},
  url          = {http://dx.doi.org/10.5281/zenodo.53936}
}
\end{filecontents*}
\documentclass[12pt]{amsart}

\usepackage{inputs/macros}

\pagestyle{fancy}
\lhead{William DeMeo} \chead{} \rhead{December 2018}
%\lfoot{} \rfoot{\bf \thepage} \cfoot{}


\newcommand\alg[1]{\mathbf{#1}}
\newcommand\defn[1]{\textit{#1}}
\newcommand\sV{\mathcal{V}}
\newcommand{\comm}[2]{\ensuremath{[#1, #2]}} % [r, s] (relation spacing)
\newcommand{\malcev}{Maltsev\xspace}


\begin{document}
\begin{center}
  {\bf Brief Summary of Research Activities}\\[4pt]
  William DeMeo
\end{center}


\vskip5mm

\section{Introduction}
The last two decades have witnessed explosive growth in the number of applications of abstract mathematics, especially in computer science, and this trend continues on a steep upward trajectory. Mathematical fields like \emph{universal algebra} and \emph{category theory} have long had a substantial influence on the development of theoretical computer science, particularly in \emph{domain theory}, \emph{denotational semantics}, and \emph{programming languages research}~\cite{MR2328298,MR1249550}. Dually, progress in theoretical computer science has informed and inspired a substantial amount of pure mathematics in the last half-century~\cite{MR3662915,MR3725758,MR2765040,MR3233442,MR1321662,MR1249550}, just as physics and physical intuition motivated so many of the mathematical discoveries of the last two centuries.  

\emph{Functional programming languages} that support \emph{dependent} and \emph{(co)inductive types} have brought about new opportunities to apply abstract concepts from universal algebra and category theory to the practice of programming, to yield code that is more modular, reusable, and safer, and to express ideas that would be difficult or impossible to express in \emph{imperative} or \emph{procedural programming languages}~\cite[Chs. 5 \& 10]{baueroplss:2018,hughes:1989,chiusano:2014}. 
The \emph{Lean Programming Language}~\cite{lean} is one example of a functional language that supports dependent and (co)inductive types, and it is ideally suited for expressing the key concepts and result of universal algebra.  As such, Lean is the primary language we have chosen for our most recently launched project, \emph{Formal Foundations for Informal Mathematics Research}.  This project is briefly mentioned in Section~\ref{sec:formal} below and described in detail in the document 
\href{https://github.com/williamdemeo/job-app/blob/master/research/demeo_informal_foundations.pdf}{demeo\_informal\_foundations.pdf}.

Universal algebra has also been invigorated by a recently discovered connection to complexity theory, and this connection was the primary focus of my work from 2015 to 2017. 
% Specifically, ``algebraic \csp theory'' is about using the tools of universal algebra to determine the computational complexity of certain decision problems. Indeed, t
The theory of finite algebras has turned out to be broadly applicable in obtaining deep and definitive results about the complexity of algorithmic problems in the broad class of  {\it constraint satisfaction problems} (\csps).   We call this area {\it algebraic \csp theory}.
The tools of universal algebra, combined with combinatorial reasoning about polymorphisms (multi-variable endomorphisms) acting on finite graphs and other relational structures, has produced deep new results concerning the complexity of \csps, and has explained and united older results. Furthermore, almost all of the new results have been turned around and used to produce
startling new algebraic results of a kind never seen before in universal algebra.

There is now an extensive and growing research literature on algebraic \csp theory. Consequently, the field of universal algebra is more active now than at any other time in its brief eighty-five year history. My work in algebraic \csp is summarized in Section~\ref{sec:csp} below. Details can be found in the manuscript entitled ``Universal Algebraic Methods for Constraint Satisfaction Problems,''~\cite{Bergman-DeMeo:2016} that I authored with Cliff Bergman, available on~\href{https://arxiv.org/abs/1611.02867}{the arXiv}.

\newpage

\null

\vskip5mm

\section{Research Projects}

\subsection{Lean and Formal Foundations}
\label{sec:formal}
I am fascinated by the connections between programming languages and mathematics, and my most recently initiated research program aims to develop a library of all core definitions and theorems of universal algebra in the Lean proof assistant and programming language~\cite{lean}.
The title of this project is \emph{Formal Foundations for Informal Mathematics Research} and a detailed project description is available in the document 
\href{https://github.com/williamdemeo/job-app/blob/master/research/demeo_informal_foundations.pdf}{demeo\_informal\_foundations.pdf}.

\subsection{New Characterizations of Bounded Lattices and Fiber Products}
\label{sec:fiber}
About a year ago I began collaborating with Peter Mayr (CU Boulder) and Nik Ruskuc (University of St. Andrews) who were interested in knowing when a homomorphism $\varphi \colon \mathbf{F} \to \mathbf{L}$ from a finitely generated free lattice $\mathbf{F}$ onto a finite lattice $\mathbf L$ has a kernel $\ker \varphi$ that is a finitely generated sublattice of $\mathbf{F}^2$.  We conjectured that this could be characterized by whether or not the homomorphism is \emph{bounded}.\footnote{See~\cite{MR1319815} for the definition of a bounded lattice homomorphism.} and I presented a proof of one direction of this conjecture at the \href{https://universalalgebra.github.io/ALH-2018/}{Algebras and Lattices in Hawaii} conference earlier this year. Last month we proved the converse and thus confirmed our conjecture.  All along Mayr and Ruskuc have had in mind an application for the fact that our new result is equivalent to a characterization of \emph{fiber products} of lattices.  With the proof of our new characterization theorem complete, we expect to have a manuscript ready for submission by January 2019.  
% In the meantime, preliminary drafts may be found in our project's \href{https://github.com/UniversalAlgebra/fg-free-lat}{github repository}.


\subsection{Tractability of Deciding Existence of Special Terms}
\label{sec:diffterm}
Among my most recent research accomplishments was the discovery that a certain decision problem about algebraic structures that previously seemed out of reach is actually tractable. In collaborative project with Ralph Freese (University of Hawaii) and Matthew Valeriote (McMaster University), we considered the following practical question: Given a finite algebra $\alg{A}$ in a finite language, can we efficiently decide whether the variety generated by $\alg{A}$ has a so called \emph{difference term}? In a paper that will soon appear in \emph{IJAC}, we answer this question (positively) in the idempotent case and then describe algorithms for constructing difference term operations~\cite{DFV:2018}.
 
A \defn{difference term} for a variety $\sV$ is a ternary term $d$ in the language of $\sV$ that satisfies the following:  if $\alg{A} = \<A, \dots \> \in \sV$, then for all $a, b \in A$ we have
  \begin{equation}
  \label{eq:3}
  d^{\alg{A}}(a,a,b) = b \quad \text{ and } \quad
  d^{\alg{A}}(a,b,b) \mathrel{\comm \theta \theta} a,
  \end{equation}
  where $\theta$ is any congruence %% of $\alg{A}$
  containing $(a,b)$
  and $[\cdot, \cdot]$ denotes the \defn{commutator}.
  %% (see Section~\ref{sec:defin-notat}).
  When the relations in (\ref{eq:3}) hold for a single algebra $\alg{A}$ and term $d$ we call $d^{\alg{A}}$
  a \defn{difference term operation} for $\alg{A}$.
  
Difference terms are studied extensively in the general algebra literature. (See, for example, \cite{MR3449235,MR1358491,MR3076179,MR1663558,KSW}.) There are many reasons to study difference terms, but
one obvious reason is the following: if we know that a variety has a difference term, this fact allows us to deduce that the algebras inhabiting this variety must satisfy certain interesting properties.  Perhaps the most important property can be summarized in the following heuristic slogan: \emph{varieties with a difference term have a commutator that behaves nicely}.  
  
The class of varieties that have a difference term is fairly broad and includes those varieties that are congruence modular or congruence meet-semidistributive. Since the commutator of two congruences of an algebra in a congruence meet-semidistributive variety is just their intersection~\cite{MR1663558}, it follows that the term $d(x,y,z) := z$ is a difference term for such varieties.  A special type of difference term $d(x,y,z)$ is one that satisfies the equations $d(x,x,y) = y$ and $d(x,y,y) = x$.  Such terms are called \emph{\malcev\ terms}. So if $\alg A$ lies in a variety that has a difference term $d(x,y,z)$ and if $\alg A$ is \emph{abelian} (i.e., $[1_A, 1_A] = 0_A$), then $d$ will be a \malcev\ term for $\alg A$.
  
Difference terms also play a role in recent work of Keith Kearnes,
Agnes Szendrei, and Ross Willard. In~\cite{MR3449235} these authors give a positive answer J\'onsson's famous question---whether a variety of finite residual bound must be finitely axiomatizable---for the
special case in which the variety has a difference term.\footnote{To say a variety has \emph{finite residual bound} is to say there is a finite bound on the size of the subdirectly irreducible members of the variety.}
  
Computers have become invaluable as a research tool and have helped to
broaden and deepen our understanding of algebraic structures and the
varieties they inhabit.  This is largely due to the efforts
of researchers who, over the last three decades, have found ingenious
ways to coax computers into solving challenging abstract algebraic
decision problems, and to do so very quickly.
To give a couple of examples related to our own work,
it is proved in~\cite{MR3239624} (respectively,~\cite{Freese:2009})
that deciding whether a finite idempotent algebra generates a variety that is congruence-$n$-permutable
(respectively, congruence-modular) is \emph{tractable}.
Our work continues this effort by presenting an efficient
algorithm for deciding whether a finitely generated idempotent variety has a difference term.


\subsection{Algebras and Algorithms, Structure and Complexity}
\label{sec:csp}
In 2015, I joined a group of eight other scientists to form arguably one of the strongest active group of researchers working in universal algebra at American universities today. Our group was awarded a three-year NSF grant for the project, ``Algebras and Algorithms, Structure and Complexity Theory.'' Our focus is on fundamental problems at the confluence of mathematical logic, algebra, and computer science, and the main goal of this effort is to deepen understanding of how to determine the complexity of certain types of computational problems.  We focus primarily on classes of mathematical problems whose solutions yield new information about the complexity of \csps. These include scheduling problems, resource allocation problems, and problems reducible to solving
systems of linear equations. \csps are theoretically solvable, but some are not solvable efficiently.  Our work seeks to establish a clear boundary between the tractable and intractable cases, and to develop efficient algorithms for solutions in the tractable cases.
My work on this project has culminated in the 50-page manuscript that I authored with Cliff Bergman entitled ``Universal Algebraic Methods for Constraint Satisfaction Problems,'' which is available on~\href{https://arxiv.org/abs/1611.02867}{the arXiv}~\cite{Bergman-DeMeo:2016}.
%% Many fundamental problems in mathematics and computer science can be
%% formulated as \csps, and progress here would have both practical and theoretical
%% significance.


%% Broadly speaking, if we achieve are able to attain the goals described above
%% will play a pivotal role in the rapidly developing area of algebraic \csp
%% theory.
%%  Some of the more specific
%% goals of the project are summarised below, but the reader may consult the
%% latest draft of our working paper~\cite{Bergman-DeMeo:2016} for more details.


%% \newpage

% 
~
\vskip5mm

\section{Structure and Complexity Theory}


%% \noindent {\bf Summary.}
%% A \acfi{CSP} asks for an assignment of
%% \acused{CSP}
%% appropriate values to variables subject to a given set of constraints.
%% Such problems arise in a very wide variety of industrial and scientific applications
%% including scheduling, resource allocation, vehicle routing, structure matching
%% in bio-molecular databases, to name just a few. As a simple example, suppose we 
%% have 50 chemical containers that must be stored in 6 sheds, and certain 
%% chemicals cannot be stored together safely. Given a list of unsafe combinations,
%% can we determine whether a solution (a safe assignment of containers to sheds)
%% exists? If so, how do we find such a solution?  Assuming we develop an algorithm
%% that solves the problem, does it scale well?  What if we are faced with 50,000
%% chemical containers? %% \\[4pt]
%  Is the solution that worked for 50 containers still  effective?
%% \noindent {\it The algebraic approach to \CSP.}

%% \section{Background}
Consider the following problems:
\begin{enumerate}
\item 
\emph{Solving a system of linear equations.}
Given $n$ linear equations of the form $a_1x_1+\cdots +a_mx_m=b$
(i.e., $n$ equations in $m$ unknowns), is there a \emph{solution},
that is, an assignment of values to the variables $x_i$
that satisfies all the equations simultaneously?

\item
  \emph{Storing chemical containers.}
  Suppose we have 6 storage sheds in which to store $500$
  chemical containers, and suppose some pairs of chemicals cannot be stored
  safely in the same shed.
  Given a list of unsafe pairs of chemicals, can we determine whether or not
  it's possible to store the containers safely?
  If so, how do we find a safe assignment of containers to sheds?
  %% If some of the containers
  %% are already safely in some of the sheds, can we leave them in place
  %% and safely put the rest in the sheds?
\end{enumerate}

In problem (1)
the goal is to assign values to variables in such a way as to satisfy 
all the \emph{equational constraints} of the system.
Problem (2) is similar in that it
also asks for an assignment of values to variables, this time subject to
\emph{relational constraints}. If we introduce a variable for every chemical container, 
the value to be assigned to a variable is the name of the storage shed
to which it is assigned, and the list of constraints on the assignment
is the list of unsafe chemical pairs.

Problems (1) and (2) are typical examples of
{\it constraint satisfaction problems} (\csps).
%% insert my version %%
A central aim of mathematical research in this area is to discover
methods for deciding when a \csp is ``easy,'' so scalable
solutions exist, and when it is ``hard,'' so scalable solutions do not exist
(unless \PeqNP).
By ``scalable solution'' we mean that the running time of the
algorithm that solves the \csp is bounded by a polynomial function of the
size of the input. (Input here refers to a particular problem instance.)
Problems for which such a ``polynomial-time'' algorithm exist belong to the 
class \P of \emph{polynomial-time-solvable} problems, and we call such
problems \emph{tractable}.

A problem is in the class \NP if there exists a polynomial-time
algorithm that takes an alleged solution to the problem and determines whether
that solution is valid.  If a problem in \NP is
at least as hard as any other problem in \NP, it is called \emph{\NPcomplete}.
We sometimes call such problems \emph{intractable}.

It is known that Problem (1), solving linear equations, is tractable.
(For example, Gaussian elimination is a polynomial-time algorithm.)
Problem (2)---the problem of deciding if it is possible to store
$n$ chemical containers in $k$ sheds (considered as a problem
with $k$ fixed and $n$ varying)---is tractable if $k\leq 2$,
while it is intractable if $k>2$.
(Readers familiar with the class of \csps known as ``graph coloring problems''
will surely have recognized that Problem~(2) is equivalent to
$k$-coloring an $n$-vertex graph.) 

%%%%%% More details and notation %%%%%%
%% More generally, each finite relational structure 
%% $\bR=\< U, R_1,\ldots,R_k\>$, $R_i\subseteq U^{n_i}$, 
%% poses an algorithmic problem
%% %, denoted 
%% $\CSP (\bR)$.  Namely, given any finite relational
%% structure $\bS=\< V,S_1,\ldots,S_k\>$ similar to  $\bR$,
%% determine if there exists a homomorphism $f: \bS \rightarrow \bR$ (and if
%% possible, find one). Each $\bS$ is an \emph{instance} of
%% the problem $\CSP (\bR)$. An instance can be recast in terms 
%% of variables and constraints.
%% Take the set of variables to be $V$, the universe of $\bS$,
%% and the constraints to be the relational
%% statements of the form $S_i(v_{j_1},\ldots, v_{n_{n_i}})$ that are 
%% true in $\bS$.
%% Then a homomorphism $\bS \to \bR$ is
%% just a mapping of all the variables to elements of $U$ that 
%% satisfies all the constraints.


%% In Problem (2) involving chemical containers the structure $\bR$
%% has universe $U$ equal to the set of storage sheds, and the only
%% relation of $\bR$ is $\neq$ (not-equals). 
%% A structure $\bS$ as in the definition of 
%% $\mbox{CSP}(\bR)$ may be interpreted
%% as having universe
%% equal to a set of chemical containers, while the binary relation
%% of $\bS$ represents those pairs of containers that
%% are unsafe in combination. A homomorphism of $\bS$ to $\bR$
%% assigns containers to sheds so that unsafe pairs of containers
%% are assigned to unequal sheds.
%%%%%% More details and notation %%%%%%

\vskip2mm

\noindent {\bf The Main Problem.}
Richard Ladner \cite{MR0464698} showed that if \PneqNP,
then there is a densely ordered set of complexity classes
%% polynomial-time reducibility classes
of problems in \NP that are neither \P nor \NP-complete.  
The \csp-{\it dichotomy conjecture} of Tom\'as Feder and 
Moshe Vardi~\cite{MR1630445} roughly states that no \csp 
belongs to such a class of ``intermediate complexity;''
i.e., every \csp is either in \P or is \NP-complete.
This conjecture has so far resisted all efforts and continues to be plausible.
%The Feder-Vardi Dichotomy Conjecture, \cite{FV99}, is 
%that every \csp is in P or is \NP-complete.
The main goal of our research in this area is to prove
the conjecture by developing methods for all classes of \csps
that can determine whether
any given \csp is in P or is \NP-complete. %% It is expected that the conjecture
%% will be proved by providing a polynomial time algorithm for solving
%% those \csps that fall into \P.

%% \newpage



~
\vskip5mm

\section{Algebras and Algorithms}

\subsection{Algebraic CSP Theory}
At issue in the study of \csps is the increase in complexity
that results when moving from a single equational or relational constraint
to a system of such constraints. It is hard to overstate the usefulness
of the ability to identify systems that can be solved quickly,
and to provide fast algorithms for their solutions. Twenty years ago
this would have been considered an impossible dream,
but in the intervening years algebraic \csp theory
has come into its own and has helped turn that dream into reality.

There is a vast and growing literature describing this development.
See for example~\cite{%
  MR2953899,
  MR2893395,
  MR3374664,
  MR3350338,
  MR2563736,
  MR2137072,
  MR2926316}.
  %% MR2648455,
Our manuscript \cite{Bergman-DeMeo:2016} also describes a subset of
these developments, as well as our latest contributions, in much greater detail
that we present here. We will touch on some of our recent accomplishments
below, but first we summarize the central
program of algebraic \csp  theory in very simple terms.

With each \csp there is an algebraic structure that corresponds to the given \csp
in a natural way. This structure has easily computable algebraic invariants, 
and the classes of algebras defined by different sets of algebraic invariants
have well defined borders. 
The known examples of \csps from different complexity classes
produce algebras that suggest an ``algebraically natural'' dividing line between
the easy and the hard \csps.
Thus, the main goal of algebraic \csp theory is to use properties
of algebras to determine the complexity class of the \csps associated with those
algebras.

This program of classifying \csps according to properties of the algebraic
structures they induce has enjoyed tremendous success over the last
decade. We now describe the program and some of its successes in more detail.
%% It has produced deep new results concerning the complexity of
%% \csps and explained and united most of the older results. 
%% Moreover, most of the new results have been turned around and used to produce
%% startling new algebraic results of a kind never seen before in universal
%% algebra.


\subsection{Reduction of the main problem}
Due to the extensive body of work cited above, the algebraic version of the
{\it \csp-dichotomy conjecture} now reduces to the following
assertion:
\begin{quote}
\emph{Every local constraint satisfaction problem associated with a finite
idempotent algebra is tractable if and only if the algebra has a Taylor term operation.}
%% \emph{If a finite idempotent algebra $\mathbf{A}$ has a
%%   so called weak near-unanimity term operation, then the associated constraint satisfaction problem 
%% $\mathrm{CSP}(\mathbf{A})$ is tractable.}
\end{quote}
%(The converse of this statement is known to be true.)

We further simplify the present discussion by focusing on a special class of algebras
called \acp{cib}. %% commutative idempotent binars (\cibs).
These are algebras with a single commutative binary operation $\cdot$
that satisfies $x \cdot x = x$ for all $x$.
For the purpose of this brief research statement, we don't need a
precise definition of Taylor term operation; it
suffices to observe that a binary operation is a Taylor term operation
%% satisfies the definition of a
%% ``weak near-unanimity''  term 
if and only if it is idempotent and commutative. This, together with 
the statement of the conjecture above, suggests the following question:
Is every finite \cib tractable?

\subsection{Recent progress}
%% The algebraic version of the dichotomy conjecture asserts that a finite algebra
%% $\mathbf A$ has a tractable constraint satisfaction problem if and only if it
%% has a weak NU term operation. The left-to-right direction is known, the converse is
%% open. 
A semilattice is an associative \cib. It has been known for quite some time that
semilattices are tractable. %% , in fact, they have finite width.
Starting from that fact, Clifford Bergman and David Failing considered various
axioms that do not imply associativity even when coupled with commutativity and
idempotence. In \cite{MR3350338}, these authors
show that a finite \cib is tractable if it satisfies an identity of 
Bol-Moufang type or the self-distributive law. 

By applying {\'A}gnes Szendrei's characterization of finite, idempotent,
strictly simple algebras in~\cite[Thm.~2.1]{MR911575}, %% Szendrei:1987}), 
we proved that every locally finite, equationally complete
variety of \cibs, except for semilattices, is congruence-permutable (hence, has
a tractable \csp).
%% Moreover, Bergman proved that such varieties are pairwise
%% independent. Consequently, the join of any two of these minimal varieties is
%% congruence-permutable. It follows from this, together with techniques
%% from~\cite{MR3350338}, that the join of any two minimal varieties of
%% \cibs has a tractable \csp.  
%% Next, Bergman and I showed that every locally finite, equationally
%% complete variety of \cibs, except for semilattices, is
%% congruence-permutable. (This follows easily from previous work of A. Szendrei.) 
%% Furthermore, Bergman showed that they are pairwise independent. Consequently, the
%% join of any two of these minimal varieties is congruence-permutable. It follows
%% from this, together with the Bergman-Failing techniques that the join of any two
%% minimal varieties of CIB's is tractable. 
Next, we observed that if $\bA$ is a finite
\cib, then the following are equivalent
(where $\mathbf S_2$ denotes the two-element semilattice):
(i) $\mathbf S_2$ is not a homomorphic image of $\bA$
(ii) $\bA$ has a so called ``cube term'' (and is thus tractable).
%% The proof is an easy application of cube-term blockers.
As a result we conclude that any variety of
\cibs that omits semilattices is tractable.  

Based on computational experiments, we were led to conjecture that if a
locally finite variety of \cibs is disjoint from semilattices, then it must be
congruence-permutable.  Keith Kearnes confirmed this conjecture using 
\cite[Lem.~2.8]{MR2333368}.
Thus, we have established that a finite \cib either is congruence permutable (and therefore
tractable) or has a proper semilattice divisor.
To settle the \csp-dichotomy conjecture for finite \cibs, then, 
it remains to show that a finite \cib with a proper semilattice  
divisor is tractable. Work on this problem continues.

To complete the initial phase of our research project, we set out to prove that all
four-element \cibs are tractable.
Using the Scala language and the \UACalc
Java libraries, we produced a list of all four-element \cibs with
two-element semilattice subalgebras.   
The program revealed that all but 32 of these algebras generate a congruence
$\mbox{SD}_\wedge$  variety, so the associated \csps are known to be solvable
by local consistency methods. 
Thus, we focused on the 32 remaining algebras,
hoping to find new polynomial-time algorithms that solve the \csps
associated with these algebras, or prove them tractable by other means.
We accomplished this goal in October of 2016 and thus settled the
\csp-dichotomy conjecture for four-element \cibs.  Our manuscript
\cite{Bergman-DeMeo:2016} presents these results; it
has been submitted for publication in {\it Logical Methods in
  Computer}, and is also available online from
\href{https://arxiv.org/abs/1611.02867}{arxiv.org/abs/1611.02867}.


%% {\bf Future Work and Collaboration.} Perhaps the most promising new idea
%% for making further progress on this problem %Petar Markovi\'{c} and 
%% is due to \textsf{Mikl\'{o}s Mar\'{o}ti}.  In unpublished notes, 
%% \textsf{Mar\'{o}ti} describes a delicate procedure for proving that certain algebras
%% yield tractable \csps. %%  The idea is quite technical, so at the risk of
%% oversimplifying I will give a very rough synopsis.  Start with a certain family
%% of algebras related in a particular way to the algebra whose tractability is
%% in question. Inductively eliminate from consideration algebras in this family
%% until we are left with .
%% The details of his approach are quite a bit more complicated than could be
%% usefully described here.  Suffice it to say that we are currently engaged in
%% an attempt to exploit this method, along with our previous results, to
%% resolve the dichotomy conjecture for \cibs.

%% four element cases, which you seem to have resolved using methods
%% similar to those we're also looking at (such as Maroti's "tree on top"
%% idea).  

%% \noindent {\bf Summary of main objectives.}
%% Understand what has been accomplished in algebraic CSP and
%% master the tools that have been developed. If that happens, we should see
%% breakthrough results in year two.

%% Our work in CSP, and in particular our most recent work with \Maroti's idea, has




%% \section*{Postscript on leveraging technology}
Our most recent work indicates that progress could be significantly accelerated
by exploiting some new developments in the programming
languages, type theory, and constructive logic communities.
For example, the Coq formal proof system~\cite{MR2229784}, is based on a version of
higher-order logic, the Calculus of inductive Constructions (CiC)~\cite{MR935892} whose
specific features---propositions as types, dependent types,
and computational reflection---seem ideally suited to the task of formalizing \csp theory,
verifying its correctness, and developing new
theorems and strategies for solving \csps. Yet
it seems there has been very little prior work on formalizing \csp theory 
using these powerful tools.

Furthermore, a number of results and ideas on which our work depends
appear in journals and conference proceedings with many important details
missing.  Using Coq, we could not only verify the correctness of the results, but
also determine the precise setting (axiomatic system and assumptions) 
required to make them valid.  To begin, some informal arguments that we
want to have formalized are those in the seminal work of Feder and Vardi mentioned
above.  Miklos Maroti has already guided students' attempts to reprove some of
these theorems in detail,  but the
proofs are far from trivial. One direction is formalizing and
certifying of theorems in Coq. Another is formalizing the algebraic
approach to \csp.  The idea is that implementing theorems as types
and proofs as ``proof objects'' would produce the following:
%% \begin{enumerate}
%% \item computer verified, and possibly simplified, proofs of known results
%% \item better understanding of existing theory and algorithms
%% \item new and/or improved theorems and algorithms
%% \end{enumerate}
(i) computer verified and possibly simplified proofs of known results,
(ii) better understanding of existing theory and algorithms, and
(iii) new and/or improved theorems and algorithms.

There is substantial evidence that Coq provides a suitable platform for
formalizing \csp.  The type theory on which Coq is based provides
a foundation for computation that is more powerful than
first-order logic and seems well suited to the specification of
constraint satisfaction problems.
%% In first-order logic, higher-level (meta) arguments are second class citizens:
%% they are interpreted as informal procedures that should be expanded to primitive
%% arguments to achieve full rigor. This is fine for informal proofs, but becomes
%% impractical in a formal one. Type theory supplies a much more satisfactory
%% solution, by providing a language that can embed meta-arguments in types. 
%% In the limited space here, we can't provide many details, so we
%% conclude by remarking that 
%% Constraints are conditions that require
%% satisfaction, which are ,
%% Moreover, the definition of `
Indeed, a \csp is most naturally specified using typed predicate calculus
expressions and explicitly requires variable-dependent sets,
just like the specification language of Martin-L\"{o}f's 
Theory of Types~\cite{MR769301}.
In order to support this feature, programming constructs should be
able to return set or type values, hence \emph{dependent types}. 
This accommodates a stronger form of function definition than is available in
most programming languages; specifically, it allows the type of a result of a
function application to depend on a formal parameter, the value (not merely the
type) of the input.

For formalizing long complex mathematical arguments Coq 
relies on \emph{computational reflection}.
Dependent types make it possible for data, functions, and \emph{potential}
computations to appear inside types. Standard mathematical practice is
to interpret and expand these objects, 
%% replacing a constant
%% by its definition, instantiating formal parameters, etc.
while Coq supports this through typing rules that lets such computation happen 
transparently. 
%% \[\frac{t : A \quad A \equiv_{\beta \iota \delta \zeta} B}{t : B}\]
%% This rule states that the 
%% $\beta \iota \delta \zeta$-computation rules
%% of CiC yield equivalent types. 
%% It is a subsumption
%% rule: there is no record of its use in the proof
%% term $t$.
Arbitrarily long computations can thus
be elided from a proof. %% , as CiC has an $\iota$-rule for
%% recursion.
This yields an entirely new way of proving
results about specific calculations.
With these tools at our disposal, we can quickly and efficiently formalize many
of our results.  Beyond merely certifying the correctness of our proofs, we hope
that this effort will result in an accelerated pace of scientific discovery.

Finally, we remark that a growing number of mathematicians, including Field's
Medalist Vladimir Voevodsky, have been using proof assistants and type theory
in their mathematics.
Constructive type theory and higher-order logics have played a noteworthy
role in mathematics, both in mechanically confirming the solution of important
problems such as the Four Color Theorem and Odd Order Theorem (Feit-Thompson
Theorem) and in settling the Kepler Conjecture.
Proof assistants have obvious potential for improving the
referee and validation process, and the type theory on which they are based may
have the potential to enrich the foundations of mathematics.
If mathematical discoveries are reaching the limits of our
ability to confirm and publish them in a timely and cost effective way, 
it is likely that proof assistants have an increasingly
important role in the mathematics of the future.
%% : computing!

%% \vskip2cm


%% %%% LEFT OFF HERE %%%%
%% Kazda believes that a better understanding of absorption can yield an algorithm
%% to test whether a finite algebra is finitely related. (This decision problem was
%% formulated by E. Aichinger.) 

%% \vskip2cm


%% {\bf Near-term goals for algebraic CSP work:} 
%% \begin{enumerate}
%% \item
%% {\bf Settle the CSP-dichotomy conjecture for finite \cibs.} To achieve this,
%% it remains to show that a finite \cib with a proper semilattice  
%% divisor is tractable. To begin with, DeMeo used the Universal Algebra Calculator to find
%% all four-element \cib that have the two-element semilattice as a subalgebra. 
%% All but 32 of these algebras generate a congruence $\mbox{SD}_\wedge$  variety
%% (so the associated \csp is known to be solvable by the ``local consistency algorithm'').
%% Therefore, we will study the 32 algebras that do not generate congruence
%% $\mbox{SD}_\wedge$ varieties, starting with the simple ones, and attempt to
%% find new polynomial-time algorithms that solve the \csps associated with
%% these algebras.
%% \end{enumerate}



%% \begin{thebibliography}{1}

%% \bibitem{BergmanFailing2013}
%% C.~Bergman and D.~Failing, \emph{Commutative, idempotent groupoids and the
%%   constraint satisfaction problem}, submitted to Algebra Universalis.

%% \bibitem{DeMeo:2014}
%% W.~DeMeo, 
%% \emph{Isotopic algebras with nonisomorphic congruence lattices},
%% Algebra Universalis \textbf{72} (2014), no.~3.

%% \bibitem{KearnesTschantz:2007}
%% K.~Kearnes and S.~Tschantz, \emph{Automorphism groups of squares and of free algebras},
%% Internat. J. Algebra Comput. \textbf{17} (2007),
%% no.~3, 461--505. %\MR{08A35 (08B20 20B25)}.


%% \bibitem{Szendrei:1987}
%% {\'A}.~Szendrei, \emph{Idempotent algebras with restrictions on subalgebras}, 
%% Acta. Sci. Math. \textbf{51} (1987), 251--268.


%% \end{thebibliography}

%% \bibitem{MarkovicMarotiMcKenzie:2012}
%% P.~Markovi{\'c} and M.~Mar{\'o}ti and R.~McKenzie,
%% \emph{Finitely related clones and algebras with cube terms},
%% Order \textbf{29} (2012), no.~2, 345--359.
%% %\MR{08A62 (08A99 08B05 08B10)}.

%% \bibitem{alvi:1987}
%% R.~McKenzie and G.~McNulty and W.~Taylor,
%% \emph{Algebras, lattices, varieties. {V}ol. {I}},
%% Wadsworth \& Brooks/Cole (1987).
%% %\MR{08-01 (06-01)}


%% \bibitem{Escardo:2008}
%% M.~Escard{\'{o}}, \emph{Exhaustible sets in higher-type
%%   computation}, Logical Methods in Computer Science \textbf{4} (2008), no.~3.

%% \bibitem{Freese:2009}
%% R.~Freese and M.~Valeriote, \emph{On the complexity of some
%%   {M}altsev conditions}, Internat. J. Algebra Comput. \textbf{19} (2009),
%%   no.~1, 41--77. \MR{2494469 (2010a:08008)}







%% \section{Project: Representing Finite Lattices}

\subsection{Introduction and Background}
Among my primary interests is the congruence lattice representation problem.
This problem lies at the intersection of \emph{group theory},
\emph{lattice theory}, and \emph{universal algebra}.
The problem of characterizing congruence lattices of finite algebras
is closely related to the problem of characterizing intervals in
the lattice of subgroups of finite groups.
We begin this section with an explanation of why congruences of an algebra
are important. Then we mention two groundbreaking results that have
deepened our understanding of the problem. Finally, 
we conclude with a description of some of our recent progress and
how we intend to push the work forward and arrive at a solution to
the main open problem.

Start with an arbitrary algebra---e.g., a group, or ring, or module;
more generally, take any set together with some finitary 
operations defined on that set, along with some equational laws that the
resulting algebra should satisfy.  Given such a structure, there
are a number of related algebras that help characterize this structure.
Among the most important of these is the lattice of congruence relations.

Congruence relations are useful in universal algebra for many reasons. Consider
the connection with specific quotients (or "homomorphic images") of
the algebra. The lattice of congruence relations of a particular algebra reveals
all the ways in which this algebra can be decomposed as a (subdirect) product of
smaller (quotient) algebras.  
Indeed, there is a one-to-one correspondence between the homomorphic images (or
quotients) and the congruence relations, so the poset of quotients of an algebra
can be identified with the congruence lattice. How does this help us?
One way is to consider the shape of the congruence lattice, which
often provides valuable information about the algebra. 

There is a deep theory of congruence lattices and what they tell us about the
underlying algebras.  The real power of congruence relations is in
characterizing varieties (equational classes) of algebras according to
properties of the congruence lattices of algebras in the variety. Much is known
about ``congruence distributive'' (CD) varieties, as well as congruence permutable
(CP) and congruence modular (CM) varieties, to name a few classes
that have been extensively studied. 

A variety is CD (CM) if every algebra in the variety has a distributive
(modular) congruence lattice. A variety is CP if the relational product
is commutative on the congruences of every algebra in the variety.
In other words, 
$\theta \circ \phi = \phi \circ \theta$ for every pair of congruences
$\theta$ and $\phi$. (We say that
$\theta$ and $\phi$ \emph{permute} in this case.)

To take a simple example, suppose we have an algebra $\bA$ and all
congruences of $\bA$ pairwise permute with one another.
(Already this can be useful for determining whether certain subdirect products
are actually direct products.)
We might then wish to check whether the whole variety generated by
$\bA$ is congruence permutable. If so, then we know that there is a ternary
function $m(x,y,z)$, built using the operations of $\bA$ and possibly
projections, that satisfies the equation $m(x,y,y)=x = m(y,y,x)$
for all $x,y\in A$. Such functions are called ``Malcev terms.''
Knowing that Malcev terms exist has proven to be enormously powerful
for things about algebras or variety of algebras. 

Examples of proofs that exploit properties of congruence lattices abound in the
literature. See, for example \cite{MR3076179} and the
2009 paper by Freese and Valeriote. The latter reveals some important
practical applications and algorithms that are now used to compute with, and
determine properties of, finite algebras were not computationally feasible
before these results were discovered.

Given an arbitrary algebra, then, it is desirable to know whether there are, 
\emph{a priori}, any restrictions on the possible shape of its congruence
lattice.  A celebrated result of Gr\"{a}tzer and
Schmidt says that there are (essentially) no such
restrictions. 
% Indeed, in~\cite{GratzerSchmidt:1963} it is
% proved that every (algebraic) lattice is the congruence lattice of some algebra.   
% A long-stading open problem in the study of such algebras is to
% characterize the class of finite lattices that are isomorphic to congruence
% lattices of finite algebras.  
% So far, it is unknown whether there are any restrictions, apart from finiteness, on
% these congruence lattices, but many suspect we will one day find an example of a finite 
% lattice that is not the congruence lattice of any finite algebra.  If we discover
% such a counterexample, we can finally declare that there are restrictions on the
% the shapes of congruence lattices, and the work to describe
% these general restrictions can begin.
Now restrict attention to \emph{finite} algebras, such as those that
come up often in applications and play important roles in computer science and
other fields.  
Given an arbitrary finite algebra, it is important to know  whether
there are any restrictions, beyond finiteness, on the shape of its congruence lattice.
It may be that there are no such restrictions and that, for every finite lattice $L$, there
exists a finite algebra whose congruence lattice is isomorphic to $L$.  
Such a lattice $L$ is called \emph{finitely representable}, and deciding whether every finite
lattice is finitely representable is known as the \emph{finite lattice
  representation problem} (\flrp).  

\subsection{Recent Accomplishments}
Over the past two years I have been experimenting with various
ways of interfacing with the Universal Algebra Calculator.
In particular, we can now write programs in languages other than Java
(e.g., Python or Scala) that exploit the UACalc Java libraries to test conjectures
about large collections of algebras.
%% , without having to manually enter each algebrainto the UACalc graphical user interface.
For example, one powerful algorithm that Ralph Freese recently implemented in Java is
an efficient method for testing whether a finite idempotent algebra generates a congruence
permutable variety. (See [26, Theorem 5.1].)
With a simple script (written in either Java, Python, or Scala) we can now apply this
method to large collections of algebras and instantly know which of these algebras generate
congruence permutable varieties.
I demonstrated this method of interacting with UACalc at the 2013 Workshop
on Computational Universal Algebra, and we are starting to see
the payoff.
Scala is a functional language with powerful facilities for concurrency and parallelism, and I
recently implemented an algorithm in Scala that uses mutually recursive functions to efficiently
search for sets of finite algebras with certain properties.

One of our current goals is to parallelize some of the more important
UACalc subroutines that, in their current serial implementation, can cause bottlenecks when using
the software, even with relatively small finite algebras. One example is the subroutine that computes
the subalgebra generated by a given set of elements. Freese has written a parallel version of
this routine and his preliminary tests show that the speedup is significant, despite the fact that
the general problem of computing a subalgebra is complete for the class of problems solvable in
polynomial time.

We close this subsection by detailing an ongoing research project in which the Universal Algebra
Calculator has been used to help locate the potentially important subcases of a problem.
William DeMeo, Ralph Freese and Peter Jipsen have used UACalc and GAP to show that all
lattices with seven or fewer elements are representable as congruence lattices of finite algebras, with
exactly one possible exception, $L_{10}$ (see Figure 1). We have also shown that a minimal
representation of $L_{10}$, if it exists, is by a transitive group (that is, by a unary algebra
whose nonconstant polynomial operations form a transitive permutation group). This means a minimal
representation of $L_{10}$ must have uniform congruences (all blocks of a congruence have the same
size). Does such a representation exist?

\todo{insert remaining part of this section (find the tex source)}

Along with my collaborators, Ralph Freese and Peter Jipsen, I discovered two new approaches to
the \flrp.
These approaches, and the planned research activities based on them,
are briefly described in the following paragraphs. For more details, please see the full
proposal.

\vskip2mm
\noindent \underline{First Approach.} A well known approach to the \flrp is % to simply
% find a representation of given finite lattice by construction -- that is,
% construct a finite algebra whose congruence lattice is isomorphic to the lattice
% in question.  This can lead to
to seek classification theorems which say that all
lattices in a certain class are representable as congruence lattices of finite
algebras.  The first approach of this proposal is of this nature, but it
involves an original method for constructing new finite algebras that provides some
control over the shape of the resulting congruence lattices.  Such algebraic
structures, called \emph{overalgebras}, are described in greater detail in
Section 4.1 of the full proposal. %, and in the article~\cite{overalgebras}.
One planned activity of the proposal
is to further investigate the overalgebra construction and, beyond merely finding
more examples of interesting algebraic constructions of this type, attempt to characterize
the class of lattices that could possibly be represented using this approach.


% {\bf (1) Intellectual Merit (1st activity):}  The first activity of our proposal
% is to further investigate overalgebra constructions and, beyond merely finding
% more examples of interesting algebraic constructions, we will attempt to characterize
% the class of lattices that can be represented using such techniques.  
% In our first paper on the subject~\cite{overalgebras}, we merely introduced the
% idea and demonstrated that such an apporach was possible, without proving any
% general classification results.  Nonetheless, this 
% already accepted for publication in {\it Algebra Universalis}, the leading
% journal in this area, ~\cite{overalgebras} .
\vskip2mm
\noindent {\bf Intellectual merit:} In the first paper on
overalgebras,\footnote{
DeMeo, W. ``Expansions of finite algebras and their congruence lattices.''
\emph{Algebra Universalis}. Available at \url{http://arxiv.org/abs/1205.1106}.}
the PI merely introduced the
idea and demonstrated that such constructions were possible, without proving any
general classification results.  Nonetheless, the paper was
accepted for publication in {\it Algebra Universalis}, the leading
journal in this area, attesting to the intellectual merit of this activity.


\vskip2mm
\noindent \underline{Second Approach.}
%The second proposed approach involves the theory of finite groups.  
That a possible
solution to the \flrp might come from group theory was discovered by %P\'eter \Palfy\ and
                                %Pavel \Pudlak\ in 1980 when, 
\Palfy\ and \Pudlak\ when in~\cite{Palfy:1980}
they proved that the following statements are equivalent:\\[4pt]
(A) Every finite lattice is isomorphic to the congruence lattice of a finite algebra.\\[4pt]
(B) Every finite lattice is isomorphic to an interval % \footnote{By an
    % \emph{interval} $[H,G]$ in a subgroup lattice, we mean the set of all subgroups of $G$
    % that contain $H$.} 
in the subgroup lattice of a finite group.\\[4pt]
Thus, an example of a finite lattice $L$ for which there is no
group having $L$ as an interval in its subgroup lattice would solve the \flrp.

\subsection{Research Planned}
%In May of 2012, 
I plan to study and classify \emph{interval enforceable properties} of finite groups.
That is, the proposed research project is to try
to characterize the properties of a finite group that are implied by
assuming the group has a given lattice as an interval in its subgroup lattice.
In order to deepen our understanding of finite groups in general,
it is natural to try to characterize those properties of a finite group that can be
inferred from the structure of an interval in its subgroup lattice.  To date, however,
very little research has been done in this direction. Given the theorem of \Palfy\ and
\Pudlak, the importance of understanding the local structure of the
subgroup lattice of a finite group, and the {\bf broader impact} of research in
this direction has become clear.
If revealing characteristics of such interval enforceable
properties of finite groups can be found, it will help to determine whether the
\flrp can be solved using the group theory approach.
% (e.g., by employing the {\it parachute lattices} introduced in the PI's
% thesis\footnote{DeMeo, W. ``Congruence lattices of finite algebras.''
% Ph.D. thesis, {U}niversity of {H}awai'i at {M}\={a}noa, Honolulu, HI (2012).})


\vskip2mm
\noindent {\bf Broader impact:} 
The \flrp is an old and important problem, and the broader impact of
progress on this problem is clear, since there is a large community of
mathematicians and other scientists who want to know the answer.
Moreover, even if the proposed research activity does not culminate in a
definitive solution to the \flrp, the classification of algebraic structures
according to the structure of their congruence lattices is a worthwhile
endeavor in its own right.  Such studies have lead to %the development of 
other important discoveries, like \emph{tame congruence theory}, which 
%initially grew out of research on the \flrp and 
%subsequently flowered 
has developed into an active and productive area, leading to
significant progress and unexpected applications, such as the 
\emph{dichotomy conjecture} of \emph{CSP}.
% \footnote{The \emph{constraint satisfaction
%     problem} (CSP) is an important area of theoretical computer science that
%   provides a common framework for many combinatorial problems in artificial
%   intelligence and applied computer science. The computational complexity and
%   approximability problems of CSP has attracted attention of researchers for a
%   long time. The most famous open problem in CSP is the
% \emph{dichotomy conjecture} of Feder and Vardi~\cite{Feder:1999}, postulating
% that every non-uniform CSP is either solvable in polynomial time
% or NP-complete, and tame congruence theory has played a major role in the
% progress made toward settling this conjecture. See, e.g.,
% \cite{BartoKozik:2009,Barto:2009,Mattetal:2010,Maroti:2008}.
% }

 %% \bibliographystyle{plain}
 %% \bibliography{inputs/refs.bib}
 \newpage

     \bibliographystyle{plainurl}
    %  \bibliography{inputs/refs2,../refs}
     \bibliography{../refs}

\end{document}
















%% \providecommand{\bysame}{\leavevmode\hbox to3em{\hrulefill}\thinspace}
%% \providecommand{\MR}{\relax\ifhmode\unskip\space\fi MR }
%% % \MRhref is called by the amsart/book/proc definition of \MR.
%% \providecommand{\MRhref}[2]{%
%%   \href{http://www.ams.org/mathscinet-getitem?mr=#1}{#2}
%% }
%% \providecommand{\href}[2]{#2}
%% \begin{thebibliography}{1}

%% \bibitem{BergmanFailing2013}
%% C.~Bergman and D.~Failing, \emph{Commutative, idempotent groupoids and the
%%   constraint satisfaction problem}, submitted to Algebra Universalis.

%% \bibitem{Maltsev1967}
%% A.~I. Mal'cev, \emph{Multiplication of classes of algebraic systems}, Siberian


%%   Math. J. \textbf{8} (1967), 254--267.

%% \bibitem{Neumann1967}
%% H.~Neumann, \emph{Varieties of groups}, Springer--Verlag, Berlin, 1967.

\def\cprime{$'$} \def\cprime{$'$} \def\cprime{$'$}
  \def\ocirc#1{\ifmmode\setbox0=\hbox{$#1$}\dimen0=\ht0 \advance\dimen0
  by1pt\rlap{\hbox to\wd0{\hss\raise\dimen0
  \hbox{\hskip.2em$\scriptscriptstyle\circ$}\hss}}#1\else {\accent"17 #1}\fi}
\providecommand{\href}[2]{#2}
\providecommand{\bysame}{\leavevmode\hbox to3em{\hrulefill}\thinspace}
\providecommand{\MR}{\relax\ifhmode\unskip\space\fi MR }
% \MRhref is called by the amsart/book/proc definition of \MR.
\providecommand{\MRhref}[2]{%
  \href{http://www.ams.org/mathscinet-getitem?mr=#1}{#2}
}
\begin{thebibliography}{1}

\bibitem{BergmanFailing2013}
C.~Bergman and D.~Failing, \emph{Commutative, idempotent groupoids and the
  constraint satisfaction problem}, submitted to Algebra Universalis.

\bibitem{Maltsev1967}
A.~I. Mal'cev, \emph{Multiplication of classes of algebraic systems}, Siberian
  Math. J. \textbf{8} (1967), 254--267.

\bibitem{Neumann1967}
H.~Neumann, \emph{Varieties of groups}, Springer--Verlag, Berlin, 1967.

\bibitem{Escardo:2008}
M.~Escard{\'{o}}, \emph{Exhaustible sets in higher-type
  computation}, Logical Methods in Computer Science \textbf{4} (2008), no.~3.

\bibitem{Freese:2009}
R.~Freese and M.~Valeriote, \emph{On the complexity of some
  {M}altsev conditions}, Internat. J. Algebra Comput. \textbf{19} (2009),
  no.~1, 41--77. \MR{2494469 (2010a:08008)}

\bibitem{KearnesTschantz:2007}
K.~Kearnes and S.~Tschantz, \emph{Automorphism groups of squares and of free algebras},
Internat. J. Algebra Comput. \textbf{17} (2007),
no.~3, 461--505 \MR{08A35 (08B20 20B25)}



Conjecture. A variety of CIB's disjoint from the variety of semilattices
is congruence permutable.

This conjecture is true. Proof follows.
-------------------------------------------------------------
-------------------------------------------------------------
Lm 2.8 from page 7 of Paper #63 on my website:
Let V be an idempotent variety that is not congruence permutable. If
F = F_V(x,y) is the 2-generated free algebra in V, then F has
subuniverses U and V such that
(1) x \in U, x \in V ,
(2) y \notin U, y \notin V , and
(3) (UxF) \cup (FxV) is a subuniverse of FxF.
-------------------------------------------------------------

Proof of the Conjecture:

Let V be a variety of CIB's that is not CP.
Let x, y, F, U, V be as in Lemma 2.8.

Case 1:
there are elements u\in U; v\in V; f, g\in F such that
u*f \notin U and g*v \notin V.
In this case, (u,g)\in UxF and (f,v)\in FxV, but the product
is not in (UxF)\cup(FxV), contradicting the fact that this union
is a subuniverse.

Thus Case 1 cannot occur, which means U*F\subseteq U or
F*V\subseteq V. Assume the former, that is, U*F\subseteq U.
Using the commutativity of the binary operation, we are
led to the alternative case

Case 2: U is an ideal. (U*F=F*U \subseteq U).

We still have x\in U and y\notin U. Since U is an ideal,
S := U\cup \{y\} is a subuniverse of F.
Since U is an ideal, the subuniverse S has a congruence
\theta with 2 nonempty blocks, U and {y}.
The quotient S/\theta is a 2-element semilattice,
which belongs to V. \\\\
































Let us recall that NP denotes the family of algorithmic problems that are polynomial-time reducible
to the digraph three-coloring problem. NP-complete consists of those problems in NP to which digraph
three-coloring can be polynomial-time reduced. P is the family of problems in NP that can be solved by
a deterministic algorithm working in polynomial time. Richard Ladner [29] showed that if P = NP then
there is a densely ordered set of polynomial-time reducibility classes of problems in NP that are neither
P nor NP-complete. The \ac{CSP}-dichotomy conjecture of Tom ́
as Feder and Moshe Vardi [18] states that no
such problem can be found in the family CSP; i.e., for each R, CSP(R) is either NP-complete, or in P.
The conjecture has resisted all efforts and continues to be plausible. The work on this conjecture has
wonderfully energized several research communities. The universal algebraic approach to this conjecture
has been especially successful, producing a rich harvest of results and insights, in algebra as well as in
complexity theory. Many conferences and workshops over the past decade, for example at Vanderbilt
(2007), the American Institute of Mathematics (2008), the Fields Institute (2011), Dagstuhl Institute
(Germany) (2012) and Banf International Research Station (2014) had or will have, as a prominent theme,
the algebraic approach.
Algebra becomes relevant through these observations of Jeavons, Krokhin and Bulatov [14]: A rela-
tional structure R = A, · · · has its relational clone Clo(R) of derived relations, which is the smallest
set of relations over A containing the given relations, the trivial relations, and closed under intersection,
concatenation (or product), permutations of variables, and projections. A universal algebra A = A, · · ·
has its operational clone Clo(A) of derived operations, which is the smallest set of operations containing
the given ones, and the trivial (projection) operations, and closed under all compositions.
The relational structure R also defines a clone of operations on A, and an algebra
A(R) = A, Poly(R)
where Poly(R) (the set of polymorphisms of R) is the set of all operations on A that respect the given
relations or, in other terms, Poly(R) is the collection of all homomorphisms R n → R (for any non-negative
integer n). The algebra A also defines a relational structure
R(A) = A, Rel(A)
where Rel(A) is the set of all relations that respect the given operations (the set of admissible relations of
A), or in other terms, is the set of all subuniverses of finite powers of A.
When A is finite, the Galois connection between relations and operations is exact. Indeed, an old and
easy result states that
Rel(A(R)) = Clo(R) and Poly(R(A)) = Clo(A), if A is finite.R. McKenzie
13
Now if R = A, R 1 , . . . , R k is as above with A finite, then it is easy to see that for every structure R =
A, S 1 , . . . , S m with {S 1 , . . . , S m } ⊆ Clo(R), CSP(R ) admits a polynomial time reduction to CSP(R).
If B is any finite algebra in the variety generated by A(R), then for every relational structure S =
B, T 1 , . . . , L u formed from relations in Rel(B), there is a polynomial time reduction of CSP(S) to CSP(R).
The algorithmic complexity of CSP(R) is thus defined by the algebra A(R), and persists as a bound on
the complexity of all CSP problems residing anywhere in the variety generated by A(R). If any “difficult”
set of relations appears among the admissible relations on some finite algebra in the variety generated by
A(R), then CSP(R) is forced to be NP-complete. On the other hand, the non-occurence of such difficult
sets is likely to force R to have some interesting polymorphisms, and we can hope to use these operations
to fashion a polynomial-time algorithm for resolving CSP(R). We now discuss at some length how these
observations have proved their worth.


































Due to a rich body of recent work in the area of algebraic CSP, the algebraic
version of the \emph{\csp-dichotomy conjecture} now boils down to 
the following assertion:
\begin{quote}
\emph{If a finite idempotent algebra $\mathbf{A}$ generates a variety with a weak near
unanimity term operation, then the associated constraint satisfaction problem 
$\mathrm{CSP}(\mathbf{A})$ is tractable.}
\end{quote}
(The converse of this statement is known to be true.)

It follows directly from the definition that a binary operation is a weak
near-unanimity operation if and only if it is idempotent and commutative. This
suggests the following question: Is every finite commutative idempotent binar
tractable? (A \emph{binar} is an algebra with a single binary basic operation.)
We denote the variety of commutative idempotent binars by \cib. 

By applying {\'A}gnes Szendrei's characterization of finite, idempotent, strictly simple
algebras in~\cite[Thm.~2.1]{MR911575}, %% Szendrei:1987}), 
Bergman and DeMeo proved that every locally finite, equationally complete
variety of \cibs, except for semilattices, is congruence-permutable (hence, has
a tractable CSP). Moreover, Bergman proved that such varieties are pairwise
independent. Consequently, the join of any two of these minimal varieties is
congruence-permutable. It follows from this, together with techniques
from~\cite{MR3350338}, that the join of any two minimal varieties of
\cibs has a tractable CSP.  


\vskip2cm




Most recently, Bergman, DeMeo, and ISU doctoral student Jiali Li have shown that
for a finite \cib, A, the following are equivalent: (i) S 2 is not a divisor of
A; (ii) the variety generated by A omits tame-congruence type 5; (iii) A has an
edge term. Here, S 2 denotes the two-element semilattice. The proof is an easy
application of cube-term blockers. As a result they conclude that any variety of
\cibs that is disjoint from semilattices is tractable.

Based on numerous computational experiments, we are led to conjecture that if a
locally finite variety of \cibs is disjoint from semilattices, then it will be
congruence-permutable. Work on that conjecture is in progress.












\vskip2cm


A second component of the project investigates classical computational problems
in algebra in order to determine whether they are algorithmically solvable.
When possible, we will develop the algorithms that solve these problems and 
incorporate them into the UACalc software, a proof assistant developed
by Ralph Freese to handle computations involving algebraic structures. 
%% The researchers shall work to decide the truth of the CSP Dichotomy Conjecture
%% of Feder and Vardi, which states that every Constraint Satisfaction Problem with
%% a finite template is solvable in polynomial time or is NP complete. They will
%% further develop the algebraic approach to CSP's by refining knowledge about
%% relations compatible with weak idempotent Maltsev conditions and about algebras
%% with finitely related clones. 
%% Another aspect of the project concerns the computable recognition of properties
%% of finite algebras connected with the varieties they generate, such as whether a
%% finite algebra with a finite residual bound is finitely axiomatizable, or
%% whether a finite algebra can serve as the algebra of character values for a
%% natural duality.
Thus, one of the more tangible outcomes of the project will be
increasingly powerful and more broadly applicable software for solving
algebra problems, both theoretical and applied.
The agenda for this part of the project includes parallelizing the important subroutines,
building in conjecture-testing and search features, adding further algorithms,
and further developing the community of users and contributors. An additional
effort to incorporate some of the latest technologies developed in the Type
Theory and Programming Lagnuages research community is also under way.



\vskip2cm


















\subsection*{Computational Universal Algebra}

Recently, William DeMeo discovered a simple way of interfacing with the
Universal Algebra Calculator (UACalc) using the Python scripting language.
Specifically, Jython is an implementation of Python that runs on the Java Virtual
Machine (JVM). This makes it possible to include UACalc Java packages
inside Python programs, so anyone who knows a little Python can write simple
scripts to test conjectures about large collections of
algebras, without having to manually enter each algebra into the UACalc
graphical user interface. 

For example, one powerful feature of the UACalc that Ralph Freese recently
implemented is an efficient method for quickly testing whether a finite
idempotent algebra generates a congruence permutable variety.  (See
\cite[Theorem 5.1]{Freese:2009}.)  With a simple Python script we can now apply
this method to a large batch of algebras and almost instantly know which ones
generate congruence permutable varieties.

DeMeo demonstrated this means of interacting with UACalc at the 2013 Workshop on
Computational Universal Algebra, and Peter Jipsen pointed out that the Scala
programming language is also based on the JVM and could be used in the same
way.  This has turned out to be a valuable insight, since Scala supports both
the functional and object oriented programming paradigms, and has powerful
facilities for concurrency and parallelism.  Recently, DeMeo implemented
an algorithm in Scala that uses mutually recursive functions to efficiently
search for sets of finite algebras with certain properties.

One of the current goals of DeMeo and Freese is to parallelize some of the more
important UACalc subroutines that, in their current serial implementation, can
cause bottlenecks when using the software, even with relatively small finite
algebras.  One example is the subroutine that computes the subalgebra generated
by a given set of elements.  Freese has written a parallel version of this
routine and his preliminary tests show that the speedup is significant, despite
the fact that the general problem of computing a subalgebra is complete for the
class of problems solvable in polynomial time.
