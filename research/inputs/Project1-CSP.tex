
~
\vskip5mm

\section{Structure and Complexity Theory}


%% \noindent {\bf Summary.}
%% A \acfi{CSP} asks for an assignment of
%% \acused{CSP}
%% appropriate values to variables subject to a given set of constraints.
%% Such problems arise in a very wide variety of industrial and scientific applications
%% including scheduling, resource allocation, vehicle routing, structure matching
%% in bio-molecular databases, to name just a few. As a simple example, suppose we 
%% have 50 chemical containers that must be stored in 6 sheds, and certain 
%% chemicals cannot be stored together safely. Given a list of unsafe combinations,
%% can we determine whether a solution (a safe assignment of containers to sheds)
%% exists? If so, how do we find such a solution?  Assuming we develop an algorithm
%% that solves the problem, does it scale well?  What if we are faced with 50,000
%% chemical containers? %% \\[4pt]
%  Is the solution that worked for 50 containers still  effective?
%% \noindent {\it The algebraic approach to \CSP.}

%% \section{Background}
Consider the following problems:
\begin{enumerate}
\item 
\emph{Solving a system of linear equations.}
Given $n$ linear equations of the form $a_1x_1+\cdots +a_mx_m=b$
(i.e., $n$ equations in $m$ unknowns), is there a \emph{solution},
that is, an assignment of values to the variables $x_i$
that satisfies all the equations simultaneously?

\item
  \emph{Storing chemical containers.}
  Suppose we have 6 storage sheds in which to store $500$
  chemical containers, and suppose some pairs of chemicals cannot be stored
  safely in the same shed.
  Given a list of unsafe pairs of chemicals, can we determine whether or not
  it's possible to store the containers safely?
  If so, how do we find a safe assignment of containers to sheds?
  %% If some of the containers
  %% are already safely in some of the sheds, can we leave them in place
  %% and safely put the rest in the sheds?
\end{enumerate}

In problem (1)
the goal is to assign values to variables in such a way as to satisfy 
all the \emph{equational constraints} of the system.
Problem (2) is similar in that it
also asks for an assignment of values to variables, this time subject to
\emph{relational constraints}. If we introduce a variable for every chemical container, 
the value to be assigned to a variable is the name of the storage shed
to which it is assigned, and the list of constraints on the assignment
is the list of unsafe chemical pairs.

Problems (1) and (2) are typical examples of
{\it constraint satisfaction problems} (\csps).
%% insert my version %%
A central aim of mathematical research in this area is to discover
methods for deciding when a \csp is ``easy,'' so scalable
solutions exist, and when it is ``hard,'' so scalable solutions do not exist
(unless \PeqNP).
By ``scalable solution'' we mean that the running time of the
algorithm that solves the \csp is bounded by a polynomial function of the
size of the input. (Input here refers to a particular problem instance.)
Problems for which such a ``polynomial-time'' algorithm exist belong to the 
class \P of \emph{polynomial-time-solvable} problems, and we call such
problems \emph{tractable}.

A problem is in the class \NP if there exists a polynomial-time
algorithm that takes an alleged solution to the problem and determines whether
that solution is valid.  If a problem in \NP is
at least as hard as any other problem in \NP, it is called \emph{\NPcomplete}.
We sometimes call such problems \emph{intractable}.

It is known that Problem (1), solving linear equations, is tractable.
(For example, Gaussian elimination is a polynomial-time algorithm.)
Problem (2)---the problem of deciding if it is possible to store
$n$ chemical containers in $k$ sheds (considered as a problem
with $k$ fixed and $n$ varying)---is tractable if $k\leq 2$,
while it is intractable if $k>2$.
(Readers familiar with the class of \csps known as ``graph coloring problems''
will surely have recognized that Problem~(2) is equivalent to
$k$-coloring an $n$-vertex graph.) 

%%%%%% More details and notation %%%%%%
%% More generally, each finite relational structure 
%% $\bR=\< U, R_1,\ldots,R_k\>$, $R_i\subseteq U^{n_i}$, 
%% poses an algorithmic problem
%% %, denoted 
%% $\CSP (\bR)$.  Namely, given any finite relational
%% structure $\bS=\< V,S_1,\ldots,S_k\>$ similar to  $\bR$,
%% determine if there exists a homomorphism $f: \bS \rightarrow \bR$ (and if
%% possible, find one). Each $\bS$ is an \emph{instance} of
%% the problem $\CSP (\bR)$. An instance can be recast in terms 
%% of variables and constraints.
%% Take the set of variables to be $V$, the universe of $\bS$,
%% and the constraints to be the relational
%% statements of the form $S_i(v_{j_1},\ldots, v_{n_{n_i}})$ that are 
%% true in $\bS$.
%% Then a homomorphism $\bS \to \bR$ is
%% just a mapping of all the variables to elements of $U$ that 
%% satisfies all the constraints.


%% In Problem (2) involving chemical containers the structure $\bR$
%% has universe $U$ equal to the set of storage sheds, and the only
%% relation of $\bR$ is $\neq$ (not-equals). 
%% A structure $\bS$ as in the definition of 
%% $\mbox{CSP}(\bR)$ may be interpreted
%% as having universe
%% equal to a set of chemical containers, while the binary relation
%% of $\bS$ represents those pairs of containers that
%% are unsafe in combination. A homomorphism of $\bS$ to $\bR$
%% assigns containers to sheds so that unsafe pairs of containers
%% are assigned to unequal sheds.
%%%%%% More details and notation %%%%%%

\vskip2mm

\noindent {\bf The Main Problem.}
Richard Ladner \cite{MR0464698} showed that if \PneqNP,
then there is a densely ordered set of complexity classes
%% polynomial-time reducibility classes
of problems in \NP that are neither \P nor \NP-complete.  
The \csp-{\it dichotomy conjecture} of Tom\'as Feder and 
Moshe Vardi~\cite{MR1630445} roughly states that no \csp 
belongs to such a class of ``intermediate complexity;''
i.e., every \csp is either in \P or is \NP-complete.
This conjecture has so far resisted all efforts and continues to be plausible.
%The Feder-Vardi Dichotomy Conjecture, \cite{FV99}, is 
%that every \csp is in P or is \NP-complete.
The main goal of our research in this area is to prove
the conjecture by developing methods for all classes of \csps
that can determine whether
any given \csp is in P or is \NP-complete. %% It is expected that the conjecture
%% will be proved by providing a polynomial time algorithm for solving
%% those \csps that fall into \P.

%% \newpage



~
\vskip5mm

\section{Algebras and Algorithms}

\subsection{Algebraic CSP Theory}
At issue in the study of \csps is the increase in complexity
that results when moving from a single equational or relational constraint
to a system of such constraints. It is hard to overstate the usefulness
of the ability to identify systems that can be solved quickly,
and to provide fast algorithms for their solutions. Twenty years ago
this would have been considered an impossible dream,
but in the intervening years algebraic \csp theory
has come into its own and has helped turn that dream into reality.

There is a vast and growing literature describing this development.
See for example~\cite{%
  MR2953899,
  MR2893395,
  MR3374664,
  MR3350338,
  MR2563736,
  MR2137072,
  MR2926316}.
  %% MR2648455,
Our manuscript \cite{Bergman-DeMeo:2016} also describes a subset of
these developments, as well as our latest contributions, in much greater detail
that we present here. We will touch on some of our recent accomplishments
below, but first we summarize the central
program of algebraic \csp  theory in very simple terms.

With each \csp there is an algebraic structure that corresponds to the given \csp
in a natural way. This structure has easily computable algebraic invariants, 
and the classes of algebras defined by different sets of algebraic invariants
have well defined borders. 
The known examples of \csps from different complexity classes
produce algebras that suggest an ``algebraically natural'' dividing line between
the easy and the hard \csps.
Thus, the main goal of algebraic \csp theory is to use properties
of algebras to determine the complexity class of the \csps associated with those
algebras.

This program of classifying \csps according to properties of the algebraic
structures they induce has enjoyed tremendous success over the last
decade. We now describe the program and some of its successes in more detail.
%% It has produced deep new results concerning the complexity of
%% \csps and explained and united most of the older results. 
%% Moreover, most of the new results have been turned around and used to produce
%% startling new algebraic results of a kind never seen before in universal
%% algebra.


\subsection{Reduction of the main problem}
Due to the extensive body of work cited above, the algebraic version of the
{\it \csp-dichotomy conjecture} now reduces to the following
assertion:
\begin{quote}
\emph{Every local constraint satisfaction problem associated with a finite
idempotent algebra is tractable if and only if the algebra has a Taylor term operation.}
%% \emph{If a finite idempotent algebra $\mathbf{A}$ has a
%%   so called weak near-unanimity term operation, then the associated constraint satisfaction problem 
%% $\mathrm{CSP}(\mathbf{A})$ is tractable.}
\end{quote}
%(The converse of this statement is known to be true.)

We further simplify the present discussion by focusing on a special class of algebras
called \acp{cib}. %% commutative idempotent binars (\cibs).
These are algebras with a single commutative binary operation $\cdot$
that satisfies $x \cdot x = x$ for all $x$.
For the purpose of this brief research statement, we don't need a
precise definition of Taylor term operation; it
suffices to observe that a binary operation is a Taylor term operation
%% satisfies the definition of a
%% ``weak near-unanimity''  term 
if and only if it is idempotent and commutative. This, together with 
the statement of the conjecture above, suggests the following question:
Is every finite \cib tractable?

\subsection{Recent progress}
%% The algebraic version of the dichotomy conjecture asserts that a finite algebra
%% $\mathbf A$ has a tractable constraint satisfaction problem if and only if it
%% has a weak NU term operation. The left-to-right direction is known, the converse is
%% open. 
A semilattice is an associative \cib. It has been known for quite some time that
semilattices are tractable. %% , in fact, they have finite width.
Starting from that fact, Clifford Bergman and David Failing considered various
axioms that do not imply associativity even when coupled with commutativity and
idempotence. In \cite{MR3350338}, these authors
show that a finite \cib is tractable if it satisfies an identity of 
Bol-Moufang type or the self-distributive law. 

By applying {\'A}gnes Szendrei's characterization of finite, idempotent,
strictly simple algebras in~\cite[Thm.~2.1]{MR911575}, %% Szendrei:1987}), 
we proved that every locally finite, equationally complete
variety of \cibs, except for semilattices, is congruence-permutable (hence, has
a tractable \csp).
%% Moreover, Bergman proved that such varieties are pairwise
%% independent. Consequently, the join of any two of these minimal varieties is
%% congruence-permutable. It follows from this, together with techniques
%% from~\cite{MR3350338}, that the join of any two minimal varieties of
%% \cibs has a tractable \csp.  
%% Next, Bergman and I showed that every locally finite, equationally
%% complete variety of \cibs, except for semilattices, is
%% congruence-permutable. (This follows easily from previous work of A. Szendrei.) 
%% Furthermore, Bergman showed that they are pairwise independent. Consequently, the
%% join of any two of these minimal varieties is congruence-permutable. It follows
%% from this, together with the Bergman-Failing techniques that the join of any two
%% minimal varieties of CIB's is tractable. 
Next, we observed that if $\bA$ is a finite
\cib, then the following are equivalent
(where $\mathbf S_2$ denotes the two-element semilattice):
(i) $\mathbf S_2$ is not a homomorphic image of $\bA$
(ii) $\bA$ has a so called ``cube term'' (and is thus tractable).
%% The proof is an easy application of cube-term blockers.
As a result we conclude that any variety of
\cibs that omits semilattices is tractable.  

Based on computational experiments, we were led to conjecture that if a
locally finite variety of \cibs is disjoint from semilattices, then it must be
congruence-permutable.  Keith Kearnes confirmed this conjecture using 
\cite[Lem.~2.8]{MR2333368}.
Thus, we have established that a finite \cib either is congruence permutable (and therefore
tractable) or has a proper semilattice divisor.
To settle the \csp-dichotomy conjecture for finite \cibs, then, 
it remains to show that a finite \cib with a proper semilattice  
divisor is tractable. Work on this problem continues.

To complete the initial phase of our research project, we set out to prove that all
four-element \cibs are tractable.
Using the Scala language and the \UACalc
Java libraries, we produced a list of all four-element \cibs with
two-element semilattice subalgebras.   
The program revealed that all but 32 of these algebras generate a congruence
$\mbox{SD}_\wedge$  variety, so the associated \csps are known to be solvable
by local consistency methods. 
Thus, we focused on the 32 remaining algebras,
hoping to find new polynomial-time algorithms that solve the \csps
associated with these algebras, or prove them tractable by other means.
We accomplished this goal in October of 2016 and thus settled the
\csp-dichotomy conjecture for four-element \cibs.  Our manuscript
\cite{Bergman-DeMeo:2016} presents these results; it
has been submitted for publication in {\it Logical Methods in
  Computer}, and is also available online from
\href{https://arxiv.org/abs/1611.02867}{arxiv.org/abs/1611.02867}.


%% {\bf Future Work and Collaboration.} Perhaps the most promising new idea
%% for making further progress on this problem %Petar Markovi\'{c} and 
%% is due to \textsf{Mikl\'{o}s Mar\'{o}ti}.  In unpublished notes, 
%% \textsf{Mar\'{o}ti} describes a delicate procedure for proving that certain algebras
%% yield tractable \csps. %%  The idea is quite technical, so at the risk of
%% oversimplifying I will give a very rough synopsis.  Start with a certain family
%% of algebras related in a particular way to the algebra whose tractability is
%% in question. Inductively eliminate from consideration algebras in this family
%% until we are left with .
%% The details of his approach are quite a bit more complicated than could be
%% usefully described here.  Suffice it to say that we are currently engaged in
%% an attempt to exploit this method, along with our previous results, to
%% resolve the dichotomy conjecture for \cibs.

%% four element cases, which you seem to have resolved using methods
%% similar to those we're also looking at (such as Maroti's "tree on top"
%% idea).  

%% \noindent {\bf Summary of main objectives.}
%% Understand what has been accomplished in algebraic CSP and
%% master the tools that have been developed. If that happens, we should see
%% breakthrough results in year two.

%% Our work in CSP, and in particular our most recent work with \Maroti's idea, has


