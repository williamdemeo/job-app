\section{Project: Representing Finite Lattices}

\subsection{Introduction and Background}
Among my primary interests is the congruence lattice representation problem.
This problem lies at the intersection of \emph{group theory},
\emph{lattice theory}, and \emph{universal algebra}.
The problem of characterizing congruence lattices of finite algebras
is closely related to the problem of characterizing intervals in
the lattice of subgroups of finite groups.
We begin this section with an explanation of why congruences of an algebra
are important. Then we mention two groundbreaking results that have
deepened our understanding of the problem. Finally, 
we conclude with a description of some of our recent progress and
how we intend to push the work forward and arrive at a solution to
the main open problem.

Start with an arbitrary algebra---e.g., a group, or ring, or module;
more generally, take any set together with some finitary 
operations defined on that set, along with some equational laws that the
resulting algebra should satisfy.  Given such a structure, there
are a number of related algebras that help characterize this structure.
Among the most important of these is the lattice of congruence relations.

Congruence relations are useful in universal algebra for many reasons. Consider
the connection with specific quotients (or "homomorphic images") of
the algebra. The lattice of congruence relations of a particular algebra reveals
all the ways in which this algebra can be decomposed as a (subdirect) product of
smaller (quotient) algebras.  
Indeed, there is a one-to-one correspondence between the homomorphic images (or
quotients) and the congruence relations, so the poset of quotients of an algebra
can be identified with the congruence lattice. How does this help us?
One way is to consider the shape of the congruence lattice, which
often provides valuable information about the algebra. 

There is a deep theory of congruence lattices and what they tell us about the
underlying algebras.  The real power of congruence relations is in
characterizing varieties (equational classes) of algebras according to
properties of the congruence lattices of algebras in the variety. Much is known
about ``congruence distributive'' (CD) varieties, as well as congruence permutable
(CP) and congruence modular (CM) varieties, to name a few classes
that have been extensively studied. 

A variety is CD (CM) if every algebra in the variety has a distributive
(modular) congruence lattice. A variety is CP if the relational product
is commutative on the congruences of every algebra in the variety.
In other words, 
$\theta \circ \phi = \phi \circ \theta$ for every pair of congruences
$\theta$ and $\phi$. (We say that
$\theta$ and $\phi$ \emph{permute} in this case.)

To take a simple example, suppose we have an algebra $\bA$ and all
congruences of $\bA$ pairwise permute with one another.
(Already this can be useful for determining whether certain subdirect products
are actually direct products.)
We might then wish to check whether the whole variety generated by
$\bA$ is congruence permutable. If so, then we know that there is a ternary
function $m(x,y,z)$, built using the operations of $\bA$ and possibly
projections, that satisfies the equation $m(x,y,y)=x = m(y,y,x)$
for all $x,y\in A$. Such functions are called ``Malcev terms.''
Knowing that Malcev terms exist has proven to be enormously powerful
for things about algebras or variety of algebras. 

Examples of proofs that exploit properties of congruence lattices abound in the
literature. See, for example \cite{MR3076179} and the
2009 paper by Freese and Valeriote. The latter reveals some important
practical applications and algorithms that are now used to compute with, and
determine properties of, finite algebras were not computationally feasible
before these results were discovered.

Given an arbitrary algebra, then, it is desirable to know whether there are, 
\emph{a priori}, any restrictions on the possible shape of its congruence
lattice.  A celebrated result of Gr\"{a}tzer and
Schmidt says that there are (essentially) no such
restrictions. 
% Indeed, in~\cite{GratzerSchmidt:1963} it is
% proved that every (algebraic) lattice is the congruence lattice of some algebra.   
% A long-stading open problem in the study of such algebras is to
% characterize the class of finite lattices that are isomorphic to congruence
% lattices of finite algebras.  
% So far, it is unknown whether there are any restrictions, apart from finiteness, on
% these congruence lattices, but many suspect we will one day find an example of a finite 
% lattice that is not the congruence lattice of any finite algebra.  If we discover
% such a counterexample, we can finally declare that there are restrictions on the
% the shapes of congruence lattices, and the work to describe
% these general restrictions can begin.
Now restrict attention to \emph{finite} algebras, such as those that
come up often in applications and play important roles in computer science and
other fields.  
Given an arbitrary finite algebra, it is important to know  whether
there are any restrictions, beyond finiteness, on the shape of its congruence lattice.
It may be that there are no such restrictions and that, for every finite lattice $L$, there
exists a finite algebra whose congruence lattice is isomorphic to $L$.  
Such a lattice $L$ is called \emph{finitely representable}, and deciding whether every finite
lattice is finitely representable is known as the \emph{finite lattice
  representation problem} (\flrp).  

\subsection{Recent Accomplishments}
Over the past two years I have been experimenting with various
ways of interfacing with the Universal Algebra Calculator.
In particular, we can now write programs in languages other than Java
(e.g., Python or Scala) that exploit the UACalc Java libraries to test conjectures
about large collections of algebras.
%% , without having to manually enter each algebrainto the UACalc graphical user interface.
For example, one powerful algorithm that Ralph Freese recently implemented in Java is
an efficient method for testing whether a finite idempotent algebra generates a congruence
permutable variety. (See [26, Theorem 5.1].)
With a simple script (written in either Java, Python, or Scala) we can now apply this
method to large collections of algebras and instantly know which of these algebras generate
congruence permutable varieties.
I demonstrated this method of interacting with UACalc at the 2013 Workshop
on Computational Universal Algebra, and we are starting to see
the payoff.
Scala is a functional language with powerful facilities for concurrency and parallelism, and I
recently implemented an algorithm in Scala that uses mutually recursive functions to efficiently
search for sets of finite algebras with certain properties.

One of our current goals is to parallelize some of the more important
UACalc subroutines that, in their current serial implementation, can cause bottlenecks when using
the software, even with relatively small finite algebras. One example is the subroutine that computes
the subalgebra generated by a given set of elements. Freese has written a parallel version of
this routine and his preliminary tests show that the speedup is significant, despite the fact that
the general problem of computing a subalgebra is complete for the class of problems solvable in
polynomial time.

We close this subsection by detailing an ongoing research project in which the Universal Algebra
Calculator has been used to help locate the potentially important subcases of a problem.
William DeMeo, Ralph Freese and Peter Jipsen have used UACalc and GAP to show that all
lattices with seven or fewer elements are representable as congruence lattices of finite algebras, with
exactly one possible exception, $L_{10}$ (see Figure 1). We have also shown that a minimal
representation of $L_{10}$, if it exists, is by a transitive group (that is, by a unary algebra
whose nonconstant polynomial operations form a transitive permutation group). This means a minimal
representation of $L_{10}$ must have uniform congruences (all blocks of a congruence have the same
size). Does such a representation exist?

\todo{insert remaining part of this section (find the tex source)}

Along with my collaborators, Ralph Freese and Peter Jipsen, I discovered two new approaches to
the \flrp.
These approaches, and the planned research activities based on them,
are briefly described in the following paragraphs. For more details, please see the full
proposal.

\vskip2mm
\noindent \underline{First Approach.} A well known approach to the \flrp is % to simply
% find a representation of given finite lattice by construction -- that is,
% construct a finite algebra whose congruence lattice is isomorphic to the lattice
% in question.  This can lead to
to seek classification theorems which say that all
lattices in a certain class are representable as congruence lattices of finite
algebras.  The first approach of this proposal is of this nature, but it
involves an original method for constructing new finite algebras that provides some
control over the shape of the resulting congruence lattices.  Such algebraic
structures, called \emph{overalgebras}, are described in greater detail in
Section 4.1 of the full proposal. %, and in the article~\cite{overalgebras}.
One planned activity of the proposal
is to further investigate the overalgebra construction and, beyond merely finding
more examples of interesting algebraic constructions of this type, attempt to characterize
the class of lattices that could possibly be represented using this approach.


% {\bf (1) Intellectual Merit (1st activity):}  The first activity of our proposal
% is to further investigate overalgebra constructions and, beyond merely finding
% more examples of interesting algebraic constructions, we will attempt to characterize
% the class of lattices that can be represented using such techniques.  
% In our first paper on the subject~\cite{overalgebras}, we merely introduced the
% idea and demonstrated that such an apporach was possible, without proving any
% general classification results.  Nonetheless, this 
% already accepted for publication in {\it Algebra Universalis}, the leading
% journal in this area, ~\cite{overalgebras} .
\vskip2mm
\noindent {\bf Intellectual merit:} In the first paper on
overalgebras,\footnote{
DeMeo, W. ``Expansions of finite algebras and their congruence lattices.''
\emph{Algebra Universalis}. Available at \url{http://arxiv.org/abs/1205.1106}.}
the PI merely introduced the
idea and demonstrated that such constructions were possible, without proving any
general classification results.  Nonetheless, the paper was
accepted for publication in {\it Algebra Universalis}, the leading
journal in this area, attesting to the intellectual merit of this activity.


\vskip2mm
\noindent \underline{Second Approach.}
%The second proposed approach involves the theory of finite groups.  
That a possible
solution to the \flrp might come from group theory was discovered by %P\'eter \Palfy\ and
                                %Pavel \Pudlak\ in 1980 when, 
\Palfy\ and \Pudlak\ when in~\cite{Palfy:1980}
they proved that the following statements are equivalent:\\[4pt]
(A) Every finite lattice is isomorphic to the congruence lattice of a finite algebra.\\[4pt]
(B) Every finite lattice is isomorphic to an interval % \footnote{By an
    % \emph{interval} $[H,G]$ in a subgroup lattice, we mean the set of all subgroups of $G$
    % that contain $H$.} 
in the subgroup lattice of a finite group.\\[4pt]
Thus, an example of a finite lattice $L$ for which there is no
group having $L$ as an interval in its subgroup lattice would solve the \flrp.

\subsection{Research Planned}
%In May of 2012, 
I plan to study and classify \emph{interval enforceable properties} of finite groups.
That is, the proposed research project is to try
to characterize the properties of a finite group that are implied by
assuming the group has a given lattice as an interval in its subgroup lattice.
In order to deepen our understanding of finite groups in general,
it is natural to try to characterize those properties of a finite group that can be
inferred from the structure of an interval in its subgroup lattice.  To date, however,
very little research has been done in this direction. Given the theorem of \Palfy\ and
\Pudlak, the importance of understanding the local structure of the
subgroup lattice of a finite group, and the {\bf broader impact} of research in
this direction has become clear.
If revealing characteristics of such interval enforceable
properties of finite groups can be found, it will help to determine whether the
\flrp can be solved using the group theory approach.
% (e.g., by employing the {\it parachute lattices} introduced in the PI's
% thesis\footnote{DeMeo, W. ``Congruence lattices of finite algebras.''
% Ph.D. thesis, {U}niversity of {H}awai'i at {M}\={a}noa, Honolulu, HI (2012).})


\vskip2mm
\noindent {\bf Broader impact:} 
The \flrp is an old and important problem, and the broader impact of
progress on this problem is clear, since there is a large community of
mathematicians and other scientists who want to know the answer.
Moreover, even if the proposed research activity does not culminate in a
definitive solution to the \flrp, the classification of algebraic structures
according to the structure of their congruence lattices is a worthwhile
endeavor in its own right.  Such studies have lead to %the development of 
other important discoveries, like \emph{tame congruence theory}, which 
%initially grew out of research on the \flrp and 
%subsequently flowered 
has developed into an active and productive area, leading to
significant progress and unexpected applications, such as the 
\emph{dichotomy conjecture} of \emph{CSP}.
% \footnote{The \emph{constraint satisfaction
%     problem} (CSP) is an important area of theoretical computer science that
%   provides a common framework for many combinatorial problems in artificial
%   intelligence and applied computer science. The computational complexity and
%   approximability problems of CSP has attracted attention of researchers for a
%   long time. The most famous open problem in CSP is the
% \emph{dichotomy conjecture} of Feder and Vardi~\cite{Feder:1999}, postulating
% that every non-uniform CSP is either solvable in polynomial time
% or NP-complete, and tame congruence theory has played a major role in the
% progress made toward settling this conjecture. See, e.g.,
% \cite{BartoKozik:2009,Barto:2009,Mattetal:2010,Maroti:2008}.
% }
