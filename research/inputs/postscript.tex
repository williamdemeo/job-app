\section*{Postscript on leveraging technology}
Our most recent work indicates that progress could be significantly accelerated
by exploiting some new developments in the programming
languages, type theory, and constructive logic communities.
For example, the Coq formal proof system~\cite{MR2229784}, is based on a version of
higher-order logic, the Calculus of inductive Constructions (CiC)~\cite{MR935892} whose
specific features---propositions as types, dependent types,
and computational reflection---seem ideally suited to the task of formalizing \csp theory,
verifying its correctness, and developing new
theorems and strategies for solving \csps. Yet
it seems there has been very little prior work on formalizing \csp theory 
using these powerful tools.

Furthermore, a number of results and ideas on which our work depends
appear in journals and conference proceedings with many important details
missing.  Using Coq, we could not only verify the correctness of the results, but
also determine the precise setting (axiomatic system and assumptions) 
required to make them valid.  To begin, some informal arguments that we
want to have formalized are those in the seminal work of Feder and Vardi mentioned
above.  Miklos Maroti has already guided students' attempts to reprove some of
these theorems in detail,  but the
proofs are far from trivial. One direction is formalizing and
certifying of theorems in Coq. Another is formalizing the algebraic
approach to \csp.  The idea is that implementing theorems as types
and proofs as ``proof objects'' would produce the following:
%% \begin{enumerate}
%% \item computer verified, and possibly simplified, proofs of known results
%% \item better understanding of existing theory and algorithms
%% \item new and/or improved theorems and algorithms
%% \end{enumerate}
(i) computer verified and possibly simplified proofs of known results,
(ii) better understanding of existing theory and algorithms, and
(iii) new and/or improved theorems and algorithms.

There is substantial evidence that Coq provides a suitable platform for
formalizing \csp.  The type theory on which Coq is based provides
a foundation for computation that is more powerful than
first-order logic and seems well suited to the specification of
constraint satisfaction problems.
%% In first-order logic, higher-level (meta) arguments are second class citizens:
%% they are interpreted as informal procedures that should be expanded to primitive
%% arguments to achieve full rigor. This is fine for informal proofs, but becomes
%% impractical in a formal one. Type theory supplies a much more satisfactory
%% solution, by providing a language that can embed meta-arguments in types. 
%% In the limited space here, we can't provide many details, so we
%% conclude by remarking that 
%% Constraints are conditions that require
%% satisfaction, which are ,
%% Moreover, the definition of `
Indeed, a \csp is most naturally specified using typed predicate calculus
expressions and explicitly requires variable-dependent sets,
just like the specification language of Martin-L\"{o}f's 
Theory of Types~\cite{MR769301}.
In order to support this feature, programming constructs should be
able to return set or type values, hence \emph{dependent types}. 
This accommodates a stronger form of function definition than is available in
most programming languages; specifically, it allows the type of a result of a
function application to depend on a formal parameter, the value (not merely the
type) of the input.

For formalizing long complex mathematical arguments Coq 
relies on \emph{computational reflection}.
Dependent types make it possible for data, functions, and \emph{potential}
computations to appear inside types. Standard mathematical practice is
to interpret and expand these objects, 
%% replacing a constant
%% by its definition, instantiating formal parameters, etc.
while Coq supports this through typing rules that lets such computation happen 
transparently. 
%% \[\frac{t : A \quad A \equiv_{\beta \iota \delta \zeta} B}{t : B}\]
%% This rule states that the 
%% $\beta \iota \delta \zeta$-computation rules
%% of CiC yield equivalent types. 
%% It is a subsumption
%% rule: there is no record of its use in the proof
%% term $t$.
Arbitrarily long computations can thus
be elided from a proof. %% , as CiC has an $\iota$-rule for
%% recursion.
This yields an entirely new way of proving
results about specific calculations.
With these tools at our disposal, we can quickly and efficiently formalize many
of our results.  Beyond merely certifying the correctness of our proofs, we hope
that this effort will result in an accelerated pace of scientific discovery.

Finally, we remark that a growing number of mathematicians, including Field's
Medalist Vladimir Voevodsky, have been using proof assistants and type theory
in their mathematics.
Constructive type theory and higher-order logics have played a noteworthy
role in mathematics, both in mechanically confirming the solution of important
problems such as the Four Color Theorem and Odd Order Theorem (Feit-Thompson
Theorem) and in settling the Kepler Conjecture.
Proof assistants have obvious potential for improving the
referee and validation process, and the type theory on which they are based may
have the potential to enrich the foundations of mathematics.
If mathematical discoveries are reaching the limits of our
ability to confirm and publish them in a timely and cost effective way, 
it is likely that proof assistants have an increasingly
important role in the mathematics of the future.
%% : computing!

%% \vskip2cm


%% %%% LEFT OFF HERE %%%%
%% Kazda believes that a better understanding of absorption can yield an algorithm
%% to test whether a finite algebra is finitely related. (This decision problem was
%% formulated by E. Aichinger.) 

%% \vskip2cm


%% {\bf Near-term goals for algebraic CSP work:} 
%% \begin{enumerate}
%% \item
%% {\bf Settle the CSP-dichotomy conjecture for finite \cibs.} To achieve this,
%% it remains to show that a finite \cib with a proper semilattice  
%% divisor is tractable. To begin with, DeMeo used the Universal Algebra Calculator to find
%% all four-element \cib that have the two-element semilattice as a subalgebra. 
%% All but 32 of these algebras generate a congruence $\mbox{SD}_\wedge$  variety
%% (so the associated \csp is known to be solvable by the ``local consistency algorithm'').
%% Therefore, we will study the 32 algebras that do not generate congruence
%% $\mbox{SD}_\wedge$ varieties, starting with the simple ones, and attempt to
%% find new polynomial-time algorithms that solve the \csps associated with
%% these algebras.
%% \end{enumerate}



%% \begin{thebibliography}{1}

%% \bibitem{BergmanFailing2013}
%% C.~Bergman and D.~Failing, \emph{Commutative, idempotent groupoids and the
%%   constraint satisfaction problem}, submitted to Algebra Universalis.

%% \bibitem{DeMeo:2014}
%% W.~DeMeo, 
%% \emph{Isotopic algebras with nonisomorphic congruence lattices},
%% Algebra Universalis \textbf{72} (2014), no.~3.

%% \bibitem{KearnesTschantz:2007}
%% K.~Kearnes and S.~Tschantz, \emph{Automorphism groups of squares and of free algebras},
%% Internat. J. Algebra Comput. \textbf{17} (2007),
%% no.~3, 461--505. %\MR{08A35 (08B20 20B25)}.


%% \bibitem{Szendrei:1987}
%% {\'A}.~Szendrei, \emph{Idempotent algebras with restrictions on subalgebras}, 
%% Acta. Sci. Math. \textbf{51} (1987), 251--268.


%% \end{thebibliography}

%% \bibitem{MarkovicMarotiMcKenzie:2012}
%% P.~Markovi{\'c} and M.~Mar{\'o}ti and R.~McKenzie,
%% \emph{Finitely related clones and algebras with cube terms},
%% Order \textbf{29} (2012), no.~2, 345--359.
%% %\MR{08A62 (08A99 08B05 08B10)}.

%% \bibitem{alvi:1987}
%% R.~McKenzie and G.~McNulty and W.~Taylor,
%% \emph{Algebras, lattices, varieties. {V}ol. {I}},
%% Wadsworth \& Brooks/Cole (1987).
%% %\MR{08-01 (06-01)}


%% \bibitem{Escardo:2008}
%% M.~Escard{\'{o}}, \emph{Exhaustible sets in higher-type
%%   computation}, Logical Methods in Computer Science \textbf{4} (2008), no.~3.

%% \bibitem{Freese:2009}
%% R.~Freese and M.~Valeriote, \emph{On the complexity of some
%%   {M}altsev conditions}, Internat. J. Algebra Comput. \textbf{19} (2009),
%%   no.~1, 41--77. \MR{2494469 (2010a:08008)}




