\begin{filecontents*}{inputs/refs.bib}
@book{Pierce:SF,
  author = {Benjamin C. Pierce and Arthur Azevedo de Amorim 
                  and Chris Casinghino and Marco Gaboardi and
                  Michael Greenberg and C\v{a}t\v{a}lin Hri\c{t}cu
                  and Vilhelm Sj\"{o}berg and Brent Yorgey},
  title = {Software Foundations},
  year = {2016},
  publisher = {Electronic textbook},
  plclub = {Yes},
  bcp = {Yes},
  keys = {poplmark,books},
  note = {Version 4.0.  \url{http://www.cis.upenn.edu/~bcpierce/sf}},
  ebook = {http://www.cis.upenn.edu/~bcpierce/sf},
  japanese = {http://proofcafe.org/sf}
}
@Misc{Shanks,
  title = 	 {Open Problems in Universal Algebra,
    a Shanks workshop at Vanderbilt University},
  month = 	 {May},
  year = 	 {2015},
  note = 	 {\url{http://www.math.vanderbilt.edu/~moorm10/shanks/}}
}
\end{filecontents*}

\documentclass[11pt]{amsart}
\usepackage[margin=1.25in]{geometry}
\usepackage[colorlinks=true,urlcolor=blue,linkcolor=blue,citecolor=black]{hyperref}
\usepackage{amsmath}
\usepackage{amssymb}
\usepackage{amsthm}
\usepackage{amscd}
\usepackage{amsfonts}
\usepackage{graphicx}%
\usepackage{fancyhdr}
\usepackage{url}
\usepackage{xspace}

\theoremstyle{plain} \numberwithin{equation}{section}
\newtheorem{theorem}{Theorem}[section]
\newtheorem{corollary}[theorem]{Corollary}
\newtheorem{conjecture}{Conjecture}
\newtheorem{lemma}[theorem]{Lemma}
\newtheorem{proposition}[theorem]{Proposition}
\theoremstyle{definition}
\newtheorem{definition}[theorem]{Definition}
\newtheorem{finalremark}[theorem]{Final Remark}
\newtheorem{remark}[theorem]{Remark}
\newtheorem{example}[theorem]{Example}
\newtheorem{question}{Question} 

\newcommand{\TA}{{\small T.A.}}
\newcommand{\webwork}{{\small WeBWork}}
\newcommand{\Xpt}{5pt}
\newcommand{\csp}{\textsc{csp}\xspace}

%% \pagestyle{fancy}\chead{Teaching Philosophy and Experience} \rhead{January 2017}
%% \lhead{{\bf William DeMeo}} \lfoot{} \rfoot{\thepage} \cfoot{}
% \pagestyle{fancy}\chead{\scshape{Teaching Philosophy}} \rhead{January 2018}
% \lhead{{\scshape William DeMeo}} \lfoot{} \rfoot{\thepage} \cfoot{}
%% \pagestyle{fancy}\chead{\scshape{Teaching Experience}} \rhead{\scshape January 2017}
%% \lhead{{\scshape William DeMeo}} \lfoot{} \rfoot{\thepage} \cfoot{}
\pagestyle{fancy}\chead{} \rhead{\scshape December 2018}
\lhead{{\scshape William DeMeo}} \lfoot{} \rfoot{\thepage} \cfoot{}


\begin{document}

\begin{center}
 {\large {\scshape Teaching Philosophy, Mentoring,}}\\ {\large {\scshape and Teaching Experience}}\\[5pt]
  {\scshape William DeMeo}
\end{center}
\thispagestyle{empty}


% \section{Teaching Philosophy}
\section{Introduction}
After five years of teaching math and computer science in college classrooms, my motivation
to teach continues to grow stronger. I have had the good fortune of working in math departments
that place a heavy emphasis on teaching, and this experience has given me 
a deeper appreciation and respect for our responsibility to pass on knowledge to
future generations. It has also taught me many lessons about how to teach well.
 
Since 2012 I have taught two math courses every semester, and have
taken every opportunity available to attend teacher
training sessions, read student evaluations, and solicit feedback from
both students and peers.  This has given me many occasions on which to
reflect on my beliefs about education in general and effective
teaching in particular.  I have come to realize that my mission as a
teacher revolves around three core principles, each of which has the
main goal of engaging the student:
\begin{enumerate}
\item {\bf leadership,} to engage students by
  demonstrating a deep appreciation and passion for the subject;
\item {\bf communication,} to engage students by connecting with them
  at an appropriate level;
\item {\bf pedagogy,} to engage students by developing a set of best
  practices that stimulate them mentally and facilitate
  understanding.
\end{enumerate}

The sections that follow consider each of these core
principles in more detail and describe how they have come to form the
 core of my teaching philosophy.


 \bigskip

 \section{Core Principles}

 \subsection{Leadership}
My passion for the subject being taught must be
apparent to the students from the first day of class. 
This is essential for establishing credibility and, more importantly,
for convincing the students that the subject is truly worthy of
their attention.
Such enthusiasm is contagious and helps motivate students to engage in
active listening, participation, and learning.
I have found that convincing students of my passion for
whatever subject I teach is actually the easiest part of my
job as an educator, and each semester my teaching evaluations reveal
that my love of the subject is both obvious and contagious.


\bigskip

\subsection{Communication}  I strive to have an open, two-way
dialog of friendly yet respectful communication in the classroom.
It is important that all students---regardless of background, race,
nationality, or gender---feel welcomed and encouraged to participate in a safe environment.
Furthermore, it is important to let students know my door is always
open for them to come to me with any issues they might encounter in
class, about math or their academic and professional lives.

To accomplish these goals, I have developed a variety of strategies
based on essential educational principles of active
learning, diversity and sensitivity training, planning,
presentation and assessment.
Some of these techniques depend vitally on our awareness of
student backgrounds, interests, and abilities, as we discuss in the
next section.


\bigskip

\subsection{Learner diversity}
As we enjoy larger numbers of intelligent and eager learners from
more and more diverse backgrounds, from minority groups, both domestic and 
international, we must be cognizant of 
both the learning style of individuals and the
cultural diversity of the whole class. Using this as our guide,
we can customize our instruction and assessment accordingly.
Since a variety of learning styles are likely
to be represented in any modern classroom,
it is important to alternate frequently between instruction and
assessment.
Without frequent learner feedback and participation in every class,
it is impossible to foster a comfortable atmosphere of 
communication and mutual understanding, and without that, we cannot
assess the level of student engagement and we cannot know the degree
to which students have processed the material covered.


\pagestyle{fancy}\chead{\scshape{Teaching Philosophy}} \rhead{January 2018}
\lhead{{\scshape William DeMeo}} \lfoot{} \rfoot{\thepage} \cfoot{}

\bigskip

\section{Pedagogy}

\smallskip

\subsection{Cycles and Iterations}
One effective method of teaching is to cycle through
various modes of interaction with the students:
presentation, participation, feedback and assessment. 
Each of these modes should itself vary throughout the class and the semester.
Presentation mode, for example, can vary across a number of
dimensions.
First, the medium of presentation should
alternate between speech/audible and written/graphical/visual to
accommodate differences of learners.

Second, the
level of detail at which material is presented should proceed along a
spectrum from general to specific.  This may depend on the subject matter,
but I find it helpful to outline and motivate the entire course 
in the first lecture.  Thereafter, roughly the first quarter of
the course may be explained and motivated.  Finally, each subsection
of the first quarter is described and motivated, still using
fairly general terms. Finally each section may be covered in detail.

Proceeding along such a path, from the general to the specific, reinforces material by reiterating the same idea at various levels of abstraction. While watching Bob Harper's lectures at CMU, I learned that this strategy has a name: ``iterative deepening.'' Besides the obvious reinforcement effect that repetition has, another positive outcome of iterative deepening is that learners leave the course with a better understanding of how each topic fits into the greater scheme of the course. It is this broad view of the course, rather than the details covered in a particular section, that is likely to remain with the students long after the course has ended.


\bigskip

\subsection{The Learner's Role} Another effective way to teach a classroom of diverse learners, is to help the students stay engaged by ``making the subject their own.''  One can accomplish this by giving them some role in determining what they will learn and how they will learn it.  If students are given the chance to arrive at an understanding of some problem or topic by relying on their own ingenuity and deriving their own methods for solving the problem, then, even if the problem remains unsolved, the student's appreciation of the problem or topic is typically much deeper than if they had listened passively to a lecture. This sort of learning can be facilitated in the classroom by providing just the right amount of guidance so that learners can reach correct conclusions.  However, determining the appropriate level of guidance depends crucially on knowing the backgrounds and abilities of the learners.

Over the past few years I have experimented with this approach to great effect.  When introducing a new topic, I often try to provide just an outline of the basic ideas, and a precise description of our objective and then ask students to think, to grow accustomed to seeking answers themselves, to not wait for more guidance or answers to come from someone else. Often I leave out basic terminology and help students develop what seems like the most natural definitions and notation that might help to reach the goal. We then embark upon a journey of discovering the theory together, by considering the class's collective contributions, and making the necessary adjustments. In some cases, the students do not need very much help to arrive at sensible and useful definitions. Before long our discussion can be honed into a more formal development of the important principles, theorems, and methods.

To give a simple example, I was recently impressed by a classroom of freshman and sophomore computer science majors who, with little guidance, had no trouble formulating set-theoretic definitions of the natural numbers, the integers, ordered pairs, and $n$-tuples, despite having had no prior exposure to set theory. The notions of equivalence relation and partial ordering also seemed quite natural to most of them. Thereafter, when the class participated in the lecture and as we solved example problems, employing the definitions that \emph{they} ``invented,'' the students seemed not only more energized than usual, but also more comfortable with the material---not surprisingly, since they had a hand in its development. Again, the goal is for students to \emph{make the subject their own}.  

This experience has given me a new perspective on effective teaching, and I have learned that with only minor adjustments in presentation style, some topics can be treated in a manner that helps students gain a deeper understanding than if they passively listen to a lecture. Furthermore, with very careful planning and time management, this approach doesn't necessarily require significantly more class time.

Providing students with the main ideas and objectives and encouraging them to develop their own methods and come to their own conclusions is not always appropriate for every topic at every level.  However, there is strong evidence suggesting that, at least at the elementary school level, a hands-off approach can have quite miraculous outcomes.\footnote{See, for example, \url{http://www.wired.com/business/2013/10/free-thinkers/}} At the college and graduate levels there is a standard curriculum that must be covered in a limited amount of time, and an entirely hands-off approach is infeasible and inappropriate.  However, the benefits of class participation can be tremendous and must not be ignored at any level.


\newpage

\pagestyle{fancy}\chead{} \rhead{\scshape December 2018}
\lhead{{\scshape William DeMeo}} \lfoot{} \rfoot{\thepage} \cfoot{}

\section{Mentoring Experience} Among my most gratifying accomplishments as an educator has been successfully engaging students in math and computer science research. The following subsections highlight some of these activities.

\newcommand{\advisee}[1]{#1}

\bigskip

\subsection{Research Experiences for Undergraduates (REU)}
In the summer of 2016 at Iowa State, Cliff Bergman and I co-mentored a bright and precocious math major from University of Rochester named \advisee{Charlotte Aten}.  It was quickly apparent that Charlotte was not satisfied with anything less than most general and elegant solution to a given problem. It was this taste for generality that attracted her to universal algebra and led her to Ames. It turned out, however, that even universal algebra was not general enough for Charlotte, so I suggested we take up category theory.  This was a good fit and we spent much of the summer working through the basics of category theory up to the Yoneda Lemma. Consequently, Charlotte managed to accomplish one of her goals, which was to work out a generalized version of Cayley's Theorem for algebraic and relational structures.  

\bigskip

\subsection{TypeFunc Research Group} \label{sec:typef-rese-group} At Iowa State I founded a research group called TypeFunc (\href{https://github.com/TypeFunc/}{github.com/TypeFunc}) which convened weekly during the academic year.  In its first year, the group consisted of four undergraduate students---\advisee{Paul Lubberstedt} (math major), \advisee{Hunter Praska} (cs major), \advisee{Hyeyoung Shin} (cs major), and \advisee{Joshua Thompson} (math major)---and the goal was to generate interest in some approachable open research problems in areas related to algebra, logic, complexity and theoretical computer science, and to guide the students' reading and research activities in these areas.


There were some student presentations in spring 2015. Thompson presented proofs of some known results; he showed that the Law of Excluded Middle (LEM) is not refutable in Intuitionistic Propositional Logic, and he proved Diaconescu's Theorem (Choice implies LEM). Lubberstedt gave a presentation on the meaning of ``judgment'' in intuitionistic logic. I gave demonstrations of the Coq proof assistant, which motivated two of the students (Praska and Shin) to learn Coq by working through ``Software Foundations'' by Pierce, et al~\cite{Pierce:SF}. Shin has since become an expert in Coq after working her way through most of the Software Foundations book, and then using the proof assistant to model a simple imperative programming language.


In late spring 2016 I taught the TypeFunc group about the Constraint Satisfaction Problem (\csp) research I had been engaged in with Cliff Bergman. I explained the algebraic approach to the so called ``\csp dichotomy conjecture.'' The students seemed fascinated by this topic and, by the end of the semester, were very eager to attend a related workshop at Vanderbilt called ``Open Problems in Universal Algebra''~\cite{Shanks}.

\bigskip

\subsection{Mentoring at the University of Colorado, Boulder} During the Spring 2018 semester I co-organized (with Keith Kearnes) a reading course based on the book \emph{Algebraic Theories}, by Jiri Adamek~\cite{MR2757312}, which was attended by a very talented graduate student named Paul Lassard.  In addition, I had the privilege of serving on the Ph.D.~preliminary exam committees for Jordan DuBeau, Ali Latfi, Athena Sparks, and Michael Wheeler, the Ph.D. thesis defense committee for Jeffrey Shriner, and the thesis defense committee for Zetong Xue.

\bigskip

\subsection{Undergraduate Thesis Advising at Iowa State University} At Iowa State, I advised and collaborated with \advisee{Joshua Thompson}, one of our brightest math majors, and a student in the Honors College. Thompson worked with me on the project of determining which algebraic structures have ``absorbing subuniverses,'' and he (along with Praska and Shin, of the TypeFunc group) accompanied Cliff Bergman and me to the Shanks Workshop on ``Open Problems in Universal Algebra'' at Vanderbilt in 2015~\cite{Shanks}.  Thompson has said this was a ``life-changing experience.''

Josh Thompson's senior thesis research was a great success and he solved the open problem that I had presented him with. Specifically, Thompson identified all proper absorbing subuniverses of the 4-element algebras that Bergman and I had been studying as part of our algebraic \csp project.

\bigskip

\subsection{Undergraduate Thesis Advising at University of South Carolina}
I advised and collaborated with \advisee{Matthew Corley}, a talented undergraduate student in the Honors College, who worked with me on a project involving nonabelian harmonic analysis of musical signals. Professor Reginald Bain (Director of the Experimental Music Studio in USC's Music Department) joined this project as a co-mentor. I guided Corley though the process of developing a formal project proposal, research schedule and budget, which we submitted through the university's grant office. The proposal earned Corley a prestigious \emph{Magellan Scholarship Grant} to support his work on the project. I gave a preliminary report on this collaboration in a talk entitled ``What does a nonabelian group sound like?''~at the MAA session \emph{At the Intersection of Mathematics and the Arts}, which took place at the Joint Math Meetings in Baltimore, January, 2014.\footnote{An abstract for the MAA talk is available at:

  \url{http://www.ams.org/amsmtgs/2160_abstracts/1096-c5-2578.pdf}

The project web page resides at \url{http://soundmath.github.io/GroupSound}.}

\bigskip

\subsection{Pi Mu Epsilon and High School Math Contests} For two years at University of South Carolina I served as a faculty mentor for Pi Mu Epsilon and the Gamecock Math Club, and served on the High School Math Contest Committee charged with creating the written portion of the South Carolina High School Math Contest exam (2013, 2014).   

\pagestyle{fancy}\chead{\scshape{Mentoring Experience}} \rhead{\scshape December 2018}
\lhead{{\scshape William DeMeo}} \lfoot{} \rfoot{\thepage} \cfoot{}

\bigskip

\subsection{Mentoring at the University of Hawaii} As a graduate student, I had the opportunity to serve as mathematics mentor to the undergraduates of the Bio-Math Program---a joint program with students from the Biology, Mathematics, and Zoology Departments.  These students worked on the project of cataloging sounds of marine life, and especially fish species, with
the goal of creating a comprehensive library of audio samples that could be used by zoologists for their research.  Since I have a great deal of experience in digital signal processing and dsp software, my primary roles were teaching the students some signal processing theory and helping them get up to speed with the software tools required to achieve their goals. 

\newpage

\pagestyle{fancy}\chead{} \rhead{\scshape December 2018}
\lhead{{\scshape William DeMeo}} \lfoot{} \rfoot{\thepage} \cfoot{}

\section{Curriculum Development}
\subsection{Lean in the Classroom} I was lucky enough to teach a graduate course in Model Theory during the Spring 2018 semester at University of Colorado, Boulder.  This year I am teaching Abstract Algebra in the Fall of 2018, and Discrete Math (which serves as an ``introduction to proofs'' class) in both the Fall and Spring semesters of the 2018/19 academic year.  My experience with these courses has convinced me that the \emph{Lean Theorem Proving Language} has an important role to play in the teaching logic and proof to undergraduates.

In the Fall 2018, I presented logic to my first- and second-year undergraduate students using Gentzen style natural deduction almost exclusively. (Of course, the students were also exposed to classical logic and truth tables, and I taught them the not-so-subtle differences between classical and constructive reasoning principles.)
Because natural deduction involves only a very small set of concrete derivation rules, and essentially no axioms, it is ideal for getting the students up and proving with first-order logic. The students seem more confident and satisfied with their proofs than the students in previous courses, when the treatment was more traditional, based entirely on set theory, truth tables, and proof sequences.

Another benefit of using Gentzen style natural deduction was that it enabled the students to better appreciate \emph{soundness} and \emph{completeness}.  It seems easier to explain these concepts to students who have some experience building proof derivation trees (that are purely syntactic) on the one hand, while on the other hand having some experience with truth values and what it means for a formula to be true in a particular model.

Evidently, Paul Taylor has witnessed a similar phenomenon, as he descirbes in~\cite{MR1694820}, where he refers to ``proof boxes'' (essentially natural deduction diagrams).
\begin{quote}
``I have seen logic introduced to first year undergraduates
both in the form of truth-assignments and using proof boxes, and firmly
believe that the box method is preferable... % As this section has shown,
it is a formal version of the way mathematicians
actually reason, even if they claim to use Boolean algebra when asked.''
\end{quote}

About half of the students in my undergraduate class were computer science majors, and most of them have had some prior programming experience. So, I decided to introduce Lean in the classroom in order to keep the students  more engaged and to further develop their logical agility and reinforce their understanding of the reciprocal roles played by the introduction and elimination rules of natural deduction. At the same time, I did not want to alienate any students who had no experience with, and possibly some aversion to, computer programming. As a compromise I introduced Lean in the classroom via team homework assignments. As it turns out, their Lean proofs were among their best proofs!  They became quite skilled at constructing formal proofs in Lean, and their confidence and understanding of the analogous paper-and-pencil proofs seemed much stronger after having worked with the proof assistant.

Other educators have reported similar successful outcomes with proof assistants in a classroom setting~\cite{nipkow:2012,pierce:2009}. Kevin Buzzard at Imperial College London has the ambitious goal of formalizing the entire undergraduate mathematics curriculum in Lean~\cite{buzzard}. These circumstances point to a future where computer-aided theorem proving tools will be routinely used, resulting in better teaching, deeper student understanding, and ultimately future generations of scientists who are more productive and reliable.

\newpage

\subsection{Colorado Lean User's Group} During my first year as a postdoc at CU Boulder, I started the \emph{Colorado Lean User's Group} with the goal of encouraging graduate students in math and computer science to learn how to conduct their work in the Lean proof assistant. This effort centered around weekly hour-long meetings in the CU \emph{\href{https://plv.colorado.edu/}{Programming Languages and Verification Lab}} conference room with about five graduate students. We used the hour to learn about and discuss Lean.  By the end of the year, some of us had gained a fair amount of expertise in Lean and we are now using the language to support our research projects.

\bigskip

\subsection{Short Course on Mathematical Foundations of Computing (MFC)} During my time as a visiting assistant professor at University of Hawaii, I started the ``Hawaii MFC Circle,'' the Hawaii Chapter of TypeFunc (see~Section~\ref{sec:typef-rese-group}). With  help from some colleagues---Dusko Pavlovic in Information and Computer Science and Bj{\o}rn Kjos-Hanssen in Math---I developed a short course on the Mathematical Foundations of Computing Science that took place in April of 2017. Modelled on the successful European summer schools in theoretical computer science, the course covered a subset of the following topics: Lambda Calculus, Type Theory, Functional Programming and Dependent Types, Automated Theorem Proving in Coq.

The enthusiastic response to the initial proposal from graduate students in the Math and Computer Science Departments---and even one student at the Institute for Astronomy---surpassed my expectations. Within the first 24 hours, over a dozen students expressed a desire to take the course, and some even said they had been waiting for this kind of offering since entering the phd program. In the end we taught about 10 students for three weeks in April.  Our Github team repository, including course materials, can be found at \href{https://github.com/TypeFunc}{github.com/TypeFunc}.

\pagestyle{fancy}\chead{\scshape{Curriculum Development}} \rhead{\scshape December 2018}
\lhead{{\scshape William DeMeo}} \lfoot{} \rfoot{\thepage} \cfoot{}

\bigskip

\bibliographystyle{alphaurl}
\bibliography{../refs}

\newpage
\pagestyle{fancy}\chead{\scshape{Teaching Experience}} \rhead{\scshape December 2018}
\lhead{{\scshape William DeMeo}} \lfoot{} \rfoot{\thepage} \cfoot{}
\newcommand\cors[2]{\textit{#1: #2}}
\newcommand\crsu[3]{\textit{\href{#3}{#1: #2}}}
% \newcommand\crsu[3]{\textit{#1: #2} (\href{#3}{\small link to course})}
\newcommand\semr[1]{\hfill #1\\[\Xpt]}
\noindent \textbf{University of Colorado, Boulder} {\small (as Burnett Meyer Instructor)}\\[2pt]
\cors{Math 2001}
     {Discrete Mathematics}
\semr{Spring 2019}
\crsu{Math 2001}
     {Discrete Mathematics}
     {https://github.com/williamdemeo/math2001-fall2018}
\semr{Fall 2018}
\crsu{Math 3140}
     {Abstract Algebra}
     {https://github.com/williamdemeo/math3140-fall2018}
\semr{Fall 2018}
\crsu{Math 6000}
     {Model Theory (graduate course)}
     {https://github.com/williamdemeo/math6000-spring2018}
\semr{Spring 2018}
\crsu{Math 2130}{Linear Algebra}
     {https://github.com/williamdemeo/math2130-spring2018}
     \semr{Spring 2018}
\cors{Math 2130}{Linear Algebra}
\semr{Fall 2017}\\[-3mm]
\textbf{University of Hawaii} {\small (as Visiting Assistant Professor)}\\[2pt]
\cors{Math 215}{Applied Calculus} \semr{Spring 2017}
\cors{Math 480}{Senior Seminar} \semr{Spring 2017}
\crsu{Math 244}
     {Calculus IV}
     {https://github.com/williamdemeo/math244-fall2016}
\semr{Fall 2016}
\crsu{Math 321}{Introduction to Advanced Math}
     {https://github.com/williamdemeo/math321-fall2016}
\semr{Fall 2016}\\[-3mm]
\textbf{Iowa State University} {\small (as Postdoctoral Associate)}\\[2pt]
\crsu{Math 317}{Linear Algebra}{https://github.com/williamdemeo/Math317-Spring2016}
\semr{Spring 2016}
\cors{Math 317}{Linear Algebra}
\semr{Fall 2015}
\crsu{Math 160}{Survey of Calculus}
     {https://github.com/williamdemeo/Math160-Fall2015}
\semr{Fall 2015}
\crsu{Math 207}{Elementary Linear Algebra}
     {https://github.com/williamdemeo/Math207-Spring2015}
\semr{Spring 2015}
\crsu{Math 165}{Calculus I}
     {https://github.com/williamdemeo/Math165-Spring2015}
     \semr{Spring 2015}
\cors{Math 301}{Abstract Algebra}
\semr{Fall 2014}
\cors{Math 165}{Calculus I}
\semr{Fall 2014}\\[-3mm]
\textbf{University of South Carolina} {\small (as Visiting Assistant Professor)}\\[2pt]
\crsu{Math 700}{Linear Algebra (graduate course)}{https://github.com/williamdemeo/Math700Spring2014}\semr{Spring 2014}
\crsu{Math 141}{Calculus I}{https://github.com/williamdemeo/Math141Spring2014}\semr{Spring 2014}
\cors{Math 374}{Discrete Structures}\semr{Fall 2013}
\cors{Math 122}{Calculus for Business and Social Sciences}\semr{Fall 2013}
\cors{Math 374}{Discrete Structures}\semr{Spring 2013}
\cors{Math 122}{Calculus for Business and Social Sciences}\semr{Spring 2013}
\cors{Math 241}{Vector Calculus}\semr{Fall 2012}
\cors{Math 122}{Calculus for Business and Social Sciences}\semr{Fall 2012}\\
\textbf{University of Hawaii} {\small (as Graduate Student Instructor)}\\
\cors{Math 371}{Probability Theory}\semr{Summer 2011}
\cors{Math 215}{Applied Calculus I }\semr{Summer 2009}
\cors{Math 100}{Mathematical Reasoning}\semr{Summer 2010}
\end{document}
